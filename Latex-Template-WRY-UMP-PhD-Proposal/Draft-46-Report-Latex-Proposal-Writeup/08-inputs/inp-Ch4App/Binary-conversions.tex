% =========================================================
\subsection{App4-Converting software codes to binary bit pulses}\label{sec:Converting-software-codes-to-binary-bits-pulses}
% =========================================================

% ==========================================================
\lstset{basicstyle=\footnotesize, numberstyle=\tiny\color{blue}, frame=single, numbers=left, firstnumber=1, stepnumber=1, numbersep=1pt, xleftmargin=2.0em, framexleftmargin=1.5em, xrightmargin=0.0em, breaklines=true, breakatwhitespace=false, breakindent=5pt, prebreak=\space, postbreak=\space }
% ==========================================================

\begin{lstlisting}[caption={App4-Converting-software-codes-to-binary-bits-pulses}, label=App4-Converting-software-codes-to-binary-bits-pulses]

// File: Binary-data-string-representation.c
// Date: Wed May  1 04:07:15 MYT 2013/2018 WRY
// Author: WRY
// Envn: Linux wruslan-rtai-bismillah 2.6.32-122-rtai #rtai SMP Tue Jul 27 12:44:07 CDT 2010 i686 GNU/Linux
// Path: /home/wruslan/Documents/WRY-Hardware-ALL/aC-Programming
//
// ========================================================
/*
// http://tigcc.ticalc.org/doc/opers.html#assign
The C language offers these bitwise and logical operators:

&  (bitwise AND)
^  (bitwise exclusive OR)
|  (bitwise inclusive OR)

&& (logical AND)
|| (logical OR)

They use the following syntax:

expr1 & expr2
expr1 ^ expr2
expr1 | expr2
expr1 && expr2
expr1 || expr2

In first three expressions, both operands must be of integral type. 
In fourth and fifth expressions, both operands must be of scalar type. 

The usual arithmetical conversions are performed on expr1 and expr2.

// http://www.cprogramming.com/tutorial/bitwise_operators.html

Thinking about Bits

The byte is the lowest level at which we can access data; there's no "bit" 
type, and we can't ask for an individual bit. In fact, we can't even perform
operations on a single bit -- every bitwise operator will be applied to, at
a minimum, an entire byte at a time. This means we'll be considering the whole
representation of a number whenever we talk about applying a bitwise operator. 

(Note that this doesn't mean we can't ever change only one bit at a time; it
just means we have to be smart about how we do it.) 

Understanding what it means to apply a bitwise operator to an entire string 
of bits is probably easiest to see with the shifting operators. By convention, 
in C and C++ you can think about binary numbers as starting with the most 
significant bit to the left (i.e., 10000000 is 128, and 00000001 is 1). 

Regardless of underlying representation, you may treat this as true. As a 
consequence, the results of the left shift (<<) and right shift (>>) operators 
are not implementation dependent for unsigned numbers (for signed numbers, the 
right shift operator is implementation defined).

The leftshift operator is the equivalent of moving all the bits of a number
a specified number of places to the left:

[variable]<<[number of places]

*/
// ========================================================
#include <stdio.h>
#include <string.h>
#include <limits.h>

char x, z;
int y;

// BINARY REPRESENTATION FOR PRINTING (DISPLAY) TO TERMINAL
char bin8_output[33] = {0};	// LAST CHAR IS NULL "\0"
char bin16_output[33] = {0};	// LAST CHAR IS NULL "\0"
char bin32_output[33] = {0};	// LAST CHAR IS NULL "\0"

// FUNCTION PROTOTYPES
void convert_integer_to_binary8 (int input_int, char *output_bin);
void convert_integer_to_binary16(int input_int, char *output_bin);
void convert_integer_to_binary32(int input_int, char *output_bin);

// ========================================================
void convert_integer_to_binary8(int input_int, char *output_bin8) {
// ========================================================
unsigned int mask8;
unsigned int mask16;
unsigned int mask32;      
// 	used to check each individual bit, 
// 	unsigned to alleviate sign extension problems

// 	Set only the high-end bit, 4-bits/byte
// 	8-bit  = 2 bytes,  mask8  = 1000 0000
// 	16-bit = 4 bytes,  mask16 = 1000 0000 0000 0000
// 	32-bit = 8 bytes,  mask32 = 1000 0000 0000 0000 0000 0000 0000 0000

//    	mask8  = 0x80; 	  
	mask8  = 0b10000000 ; 
//  	mask16 = 0x8000; 
	mask16 = 0b1000000000000000;	    
//  	mask32 = 0x80000000;    
	mask32 = 0b10000000000000000000000000000000;

	char *masked8[9]={"", "10000000","01000000","00100000", "00010000", \
			      "00001000","00000100","00000010", "00000001"};

	printf("INTEGER INPUT = %d \n", input_int);
	int bitposition = 0;    
	while (mask8)  {         		 // Loop until MASK is empty

		bitposition++;
		if (input_int & mask8) {     	 // Bitwise AND => test the masked bit
			*output_bin8 = '1';      // if true, binary value 1 is appended to output array

			/*  DEBUGGING              
			printf("bitposition = %d \tTRUE \tmask8(BIN, DEC, HEX) = (%s, %d, %x) ", \
			bitposition, masked8[bitposition], mask8, mask8);
			printf("\toutput_bin8 = %s \n", output_bin8);
			*/
		} else {
			*output_bin8 = '0';     // if false, binary value 0 is appended to output array
	
			/* DEBUGGING	      	
				printf("bitposition = %d \tFALSE \tmask8(BIN, DEC, HEX) = (%s, %d, %x) ", \
				bitposition, masked8[bitposition], mask8, mask8);
				printf("\toutput_bin8 = %s \n", output_bin8);
			*/
		} END if .. else

		output_bin8++;               // next character
		mask8 >>= 1;                 // shift the mask variable 1 bit to the right
		
	} // END while loop
	
	*output_bin8 = 0;                 	// add the trailing null to return value 
	
} // END void function()

// ========================================================
void convert_integer_to_binary16(int input_int, char *output_bin16) {
// ========================================================
unsigned int mask16;
mask16 = 0x8000;

printf("INTEGER INPUT = %d \n", input_int);

char *masked16[17]={"", "1000000000000000", "0100000000000000", "0010000000000000", "0001000000000000", \
			"0000010000000000", "0000010000000000", "0000001000000000", "0000000100000000", \
			"0000000010000000", "0000000001000000", "0000000000100000", "0000000000010000", \
			"0000000000001000", "0000000000000100", "0000000000000010", "0000000000000001"};

	int bitposition = 0;    
	while (mask16)  {         // Loop until MASK is empty

		bitposition++;
		if (input_int & mask16) {     // Bitwise AND => test the masked bit

			*output_bin16 = '1';      // if true, binary value 1 is appended to output array
		
			/* DEBUG               
			printf("bitposition = %d \tTRUE \tmask16(BIN, DEC, HEX) = (%s, %d, %x) ", \
			bitposition, masked16[bitposition], mask16, mask16);
			printf("\toutput_bin16 = %s \n", output_bin16);
			*/
		} else {
			*output_bin16 = '0';     // if false, binary value 0 is appended to output array

			/* DEBUG	      	
			printf("bitposition = %d \tFALSE \tmask16(BIN, DEC, HEX) = (%s, %d, %x) ", \
			bitposition, masked16[bitposition], mask16, mask16);
			printf("\toutput_bin16 = %s \n", output_bin16);
			*/
		} END if ..else

		output_bin16++;               // next character
		mask16 >>= 1;                 // shift the mask variable 1 bit to the right
	} // END while .. loop

	*output_bin16 = 0;                 // add the trailing null to return
} 

// ========================================================
void convert_integer_to_binary32(int input_int, char *output_bin32) {
// ========================================================
unsigned int mask32;      
mask32 = 0x80000000;    

printf("INTEGER INPUT = %d \n", input_int);

char *masked32[33]={"", "1000000000000000", "0100000000000000", "0010000000000000", "0001000000000000", \
			"0000010000000000", "0000010000000000", "0000001000000000", "0000000100000000", \
			"0000000010000000", "0000000001000000", "0000000000100000", "0000000000010000", \
			"0000000000001000", "0000000000000100", "0000000000000010", "0000000000000001"};


	int bitposition = 0;    
	while (mask32)  {         // Loop until MASK is empty

		bitposition++;
		if (input_int & mask32) {     // Bitwise AND => test the masked bit
			*output_bin32 = '1';     // if true, binary value 1 is appended to output array

			/* DEBUG               
			printf("bitposition = %d \tTRUE \tmask32(DEC, HEX) = (%d, %x) ", \
			bitposition, mask32, mask32);
			printf("\toutput_bin32 = %s \n", output_bin32);
			*/
			
		} else {
			*output_bin32 = '0';     // if false, binary value 0 is appended to output array
			
			/* DEBUG
			printf("bitposition = %d \tFALSE \tmask32(DEC, HEX) = (%d, %x) ", \
			bitposition, mask32, mask32);
			printf("\toutput_bin32 = %s \n", output_bin32);
			*/
		} // END if .. else

		output_bin32++;                // next character
		mask32 >>= 1;                  // shift the mask variable 1 bit to the right
		
	} // END while .. loop
	
*output_bin32 = 0;                 // add the trailing null to return value
}
// ========================================================
void main(int argc, char* argv[]) {
// ========================================================
printf("\nBismillah.\n\n");

	// READ INTEGER FROM TERMINAL
	printf("Input a positive integer to convert to binary. Maximum = 2147483647: ");
	scanf("%d", &y);  	// store an int as y

	// y = 2147483647;  // MAXIMUM FOR SIGNED INTEGER
	printf("\nINT_MAX = %d \tINT_MIN = %d \n", INT_MAX, INT_MIN);
	printf("CHAR_MAX = %d \tCHAR_MIN = %d \n\n", CHAR_MAX, CHAR_MIN);

	if (y >= 1 && y <=255) {
			convert_integer_to_binary8(y, bin8_output);	// AN EXECUTION NOT AN ASSIGNMENT, 
			printf("BINARY OUTPUT = %s \n", bin8_output);  // CORRECT
			printf("\n");
		
	} else if (y >= 256 && y <=65535) {
			convert_integer_to_binary16(y, bin16_output);	// AN EXECUTION NOT AN ASSIGNMENT, 
			printf("BINARY OUTPUT = %s \n", bin16_output);  // CORRECT
			printf("\n");
	
	} else if (y >= 65536 && y <=2147483647) {
			convert_integer_to_binary32(y, bin32_output);	// AN EXECUTION NOT AN ASSIGNMENT, 
			printf("BINARY OUTPUT = %s \n", bin32_output);  // CORRECT
			printf("\n");
	
	} else  if (y > 2147483647) {
			printf("Invalid positive integer input. Maximum = 2147483647\n");
	
	} else  {
			printf("Invalid positive integer input. Maximum = 2147483647\n");
	}

printf("\nAlhamdulillah.\n\n");
return(0);
}
// =========================================================
// COMPILATION AND EXECUTION
// =========================================================
/*
COMPILATION
================
wruslan@HP-ELBook8470p-ub1604-64b:~/Downloads/Temp/temp1$ gcc -o test1-convert-binary.x test1-convert-binary.c 
wruslan@HP-ELBook8470p-ub1604-64b:~/Downloads/Temp/temp1$ ls -al
....
-rw-rw-r-- 1 wruslan wruslan   6928 Dec 13 00:01 test1-convert-binary.c
-rwxrwxr-x 1 wruslan wruslan  13280 Dec 13 00:02 test1-convert-binary.x
wruslan@HP-ELBook8470p-ub1604-64b:~/Downloads/Temp/temp1$ 

EXECUTION
================
wruslan@HP-ELBook8470p-ub1604-64b:~/Downloads/Temp/temp1$ ./test1-convert-binary.x 
Bismillah.

Input a positive integer to convert to binary. Maximum = 2147483647: 255

INT_MAX = 2147483647 	INT_MIN = -2147483648 
CHAR_MAX = 127 	CHAR_MIN = -128 

INTEGER INPUT = 255 
BINARY OUTPUT = 11111111 

Alhamdulillah.
wruslan@HP-ELBook8470p-ub1604-64b:~/Downloads/Temp/temp1$ 

===============================================
COLLECTING RESULTS ONLY

INTEGER INPUT = 14 BINARY OUTPUT = 00001110 
INTEGER INPUT = 15 BINARY OUTPUT = 00001111 
INTEGER INPUT = 16 BINARY OUTPUT = 00010000 

INTEGER INPUT = 254 BINARY OUTPUT = 11111110 
INTEGER INPUT = 255 BINARY OUTPUT = 11111111 
INTEGER INPUT = 256 BINARY OUTPUT = 0000000100000000 

INTEGER INPUT = 510 BINARY OUTPUT = 0000000111111110 
INTEGER INPUT = 511 BINARY OUTPUT = 0000000111111111 
INTEGER INPUT = 512 BINARY OUTPUT = 0000001000000000 

INTEGER INPUT = 1022 BINARY OUTPUT = 0000001111111110 
INTEGER INPUT = 1023 BINARY OUTPUT = 0000001111111111 
INTEGER INPUT = 1024 BINARY OUTPUT = 0000010000000000

INTEGER INPUT = 2147483646 BINARY OUTPUT = 01111111111111111111111111111110 
INTEGER INPUT = 2147483647 BINARY OUTPUT = 01111111111111111111111111111111 

INTEGER INPUT = 2147483648
Invalid positive integer input. Maximum = 2147483647
*/
// ========================================================
\end{lstlisting}
%% =========== END LISTING =================================
