\titleformat {\chapter} {\normalfont\huge\bfseries\color{black}}   {\thechapter}{10pt}{\huge} 
\chapter {Conclusion}

%% Bismillah Hirrahma Nirrahim

The focus of this research is CNC interpolation. Technically, CNC interpolation is about the task of generating and sending signal pulses to the respective electric motors at the machine's axis-of-motions, in a coordinated and timely manner, such that the machine tool accurately tracks the desired contour path in the move from the current position to the next position. 
\vspace{0.5cm}

To achieve the task, the interpolator must generate the right number of signal pulses that achieves the desired distance for the move, and drive the correct pulse feedrate (frequency) that achieves the desired velocity for the move.	
\vspace{0.5cm}

This research emphasizes the adoption of effective and efficient software engineering technologies in the implementation details of CNC interpolation. The adoption includes sophisticated data structures, efficient process and data flow algorithms, high resolution realtime monitoring, true parallel executions, thread-safe process designs, high speed data dumps (captures), signal/slot programming mechanism and many more deployment techniques that can be creatively developed with modern software compilers available today. 
\vspace{0.5cm}

Software program control paradigms considered for the CNC Control Loop include process-driven, interrupt-driven, target-driven and optimal-driven methods. Software integration strategies considered include the utilization of back end packages and libraries like Scilab-NURBS, Octave-NURBS, Rust low-level thread safe constructs, Julia and Python parallel libraries over and above the C/C++ base program codes. This program control is the core activity for this research. 
\vspace{0.5cm}

All software implemented in this research are open source, cost-free, and actively supported by the software community. This facilitates sharing of knowledge, software solutions, programming tricks and passionate lifelong learning.
\vspace{0.5cm}

For performance comparisons, hardware architectures comprising both 64bit and 32bit systems with different configurations will be used. In addition, three different pulse generator devices will be deployed in this research project. 
\vspace{0.5cm}

The ultimate goal is to suitably implement selected software technologies that achieves our primary research target, that is, a realtime and parallel look-ahead control and feedrate compensation for the CNC reference-pulse interpolation task. For this research, the other target is to produce two conference papers and a journal paper.
\vspace{0.5cm}

This proposal is an exciting and exhilarating research project, especially for a software engineer. However, success can be elusive. On that note, it is our firm belief that the eventual success of this research rests on the help and inspiration from Allah, the All Mighty, the Most Gracious and Most Merciful. In this endeavor, we pray to Allah for his continuous blessings.
\vspace{0.5cm}

%% ==============================================	