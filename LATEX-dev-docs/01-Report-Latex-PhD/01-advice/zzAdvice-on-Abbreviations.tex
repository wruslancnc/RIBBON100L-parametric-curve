\pagebreak
%% ===========================================
\begin{tcolorbox}

\section{zzAdvice-on-Abbreviations}

\textbf{ABBREVIATIONS HOW TO Guidelines - TO REMOVE LATER}	
% \vspace*{1\baselineskip}

% ==========================================================
\lstset{basicstyle=\footnotesize, numberstyle=\tiny\color{blue}, frame=single, numbers=left, firstnumber=1, stepnumber=1, numbersep=1pt, xleftmargin=2.0em, framexleftmargin=1.5em, xrightmargin=0.0em, breaklines=true, breakatwhitespace=false, breakindent=5pt, prebreak=\space, postbreak=\space }
% ==========================================================
\begin{lstlisting}[caption={zzAdvice-on-Abbreviations}, label=zzAdvice-on-Abbreviations]
	
% (1) INFO FOR ABBREVIATIONS USE GLOSSARIES

%% PACKAGES THAT MUST BE LOADED BEFORE glossaries are hyperref, babel, polyglossia, inputenc and fontenc.

%% USE EXTERNAL TEX FILE TO ADD INTO Abbreviation-List.TEX

%% Place \usepackage{glossaries} and \makeglossaries in your preamble. The glossaries package supports multiple glossaries, acronyms, and symbols. This glossaries package replaces the glossary package.

% (2) PLACED INSIDE PREAMBLE SECTION

%% USING glossaries-extra.sty
	\usepackage[nonumberlist, acronym, automake=true]{glossaries-extra}
	\setabbreviationstyle{long-short-sc}

%% COMMAND FOR LOADING OF EXTERNAL ABBREVIATION TEX FILE
	\loadglsentries[main]{Abbreviation-List}  
	\makeglossaries   %% GIVES SORTED LIST

% (3) PLACED INSIDE DOCUMENT SECTION

%% FOR AUTO SORTED ABBREVIATION, THE SETTING IN GLOSSARIES 
%% FIRST MUST RUN PROVIDED BASH SCRIPT IN TERMINAL
%% RUN: makeglossaries "Draft-29-PhD-Proposal-WRY"

%% HERE IS LOCATION IN DOCUMENT SECTION WHERE Appendix WILL BE PRINTED
	\printglossary[title=Abbreviations, toctitle=Abbreviations] \glsaddall

% (4) EXAMPLES 

%% CONTENT LINES IN EXTERNAL Abbreviation-List.TEX FILE

%% \newabbreviation{svm}{SVM}{Support Vector Machine}
%% \newabbreviation{cnc}{CNC}{Computer Numerical Control}
%% \newabbreviation{svm}{CAD}{Computer Aided Design}

....
....
\end{lstlisting}
\end{tcolorbox}
