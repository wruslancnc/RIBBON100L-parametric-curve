%% \titleformat {\chapter} {\normalfont\huge\bfseries\color{black}}   {\thechapter}{10pt}{\huge} 
%%\chapter {Simulation Results}

	
\section{The Parametric Equations}


The ten(10) 2D parametric curves covered in this work are:
\begin{enumerate}
	\item Teardrop
	\item Butterfly
	\item Ellipse
	\item Skewed-Astroid
	\item Circle
	\item AstEpi = Astroid + Epicycloid combination
	\item Snailshell
	\item SnaHyp = Snailshell + Hypotrocoid combination
	\item Ribbon-10L
	\item Ribbon-100l = 10 times scaleup of Ribbon-10L
	
\end{enumerate}

The parametric equations describing each of the curves x(u), and y(u) are provided in the next table. The independent parameter u is limited to
\begin{equation}
u  \in  [0.0, 1.0] \nonumber
\end{equation}

The curves were selected based on their different characteristics like closed loop curves, open ended curves, symmetric or non-symmetric about the x-axis and y-axis, and having concave or convex turns. The x and y dimensions (sizes) vary among the different curves. \vspace*{1\baselineskip}

The main objective of the selection criteria is to ensure that the interpolation algorithm for the parametric curve succeeds and does not break in all cases. \vspace*{1\baselineskip}
	
The results for the feedrates in machining the ten(10) curves show continuity, smoothness, with no abrupt jumps as the CNC machine traverse the entire curve from the start (u = 0.0) until the end (u = 1.0).	\vspace*{1\baselineskip}




% ================== END CHAPTER-5 ========================