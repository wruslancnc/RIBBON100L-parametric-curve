%% File: Glossary-List.tex
%% Date: Sat Dec 15 22:26:10 +08 2018
%% Author: WRY
%% RUN BASH SCRIPT: Everytime you add or remove a glossary item.
%% ERROR: If you remove a glossary item which is still being used in the main document.

%% =========================================================
%% TYPE = MAIN ===> FOR NEW GLOSSARY ENTRY
%% =========================================================

\newglossaryentry{cnc-interpreter}{
	type={main},
	name={CNC Interpreter},
	description={The CNC INTERPRETER is xxx An Application Programming Interface (API) is a particular set
		of rules and specifications that a software program can follow to access and
		make use of the services and resources provided by another particular software
		program that implements that API
	} 
}

\newglossaryentry{cnc-interpolator}{
	type={main},
	name={CNC Interpolator},
	description={The CNC INTERPOLATOR is xxx An Application Programming Interface (API) is a particular set
		of rules and specifications that a software program can follow to access and
		make use of the services and resources provided by another particular software
		program that implements that API
	}
}

\newglossaryentry{cnc-control-loop}{
	type={main},
	name={CNC Control Loop},
	description={
		The CNC CONTROL LOOP is xxx An Application Programming Interface (API) is a particular set
		of rules and specifications that a software program can follow to access and
		make use of the services and resources provided by another particular software
		program that implements that API} 
	}

\newglossaryentry{apig}{
	type={main},
	name={API TESTING 1},
	description={
		An Application Programming Interface (API) is a particular set
		of rules and specifications that a software program can follow to access and
		make use of the services and resources provided by another particular software
		program that implements that API} 
	}

\newglossaryentry{apug}{
	type={main},
	name={APU TESTING 2},
	description={
		An Application Programming Interface (API) is a particular set
		of rules and specifications that a software program can follow to access and
		make use of the services and resources provided by another particular software
		program that implements that API} 
	}

%% =========================================================
%% TYPE = ACRONYM ===> FOR NEW GLOSSARY ENTRY
%% =========================================================
%% define the acronym and use the see= option
%% \newglossaryentry{api} {
%%	type={acronym}, 
%%	name={API}, 
%%	description={Application Programming Interface}, 
%%	first={Application Programming Interface (API)\glsadd{apig}} see=[Glossary:]{apig} }


%% define the acronym and use the see= option
%% \newglossaryentry{apui} {
%%	type={acronym}, 
%%	name={APUI}, 
%%	description={Application Programming Interface}, 
%%	first={Application Programming Interface (API)\glsadd{apig}} see=[Glossary:]{apig} }

%% =========================================================
%% ALHAMDULILLAH 3 TIMES
%% =========================================================