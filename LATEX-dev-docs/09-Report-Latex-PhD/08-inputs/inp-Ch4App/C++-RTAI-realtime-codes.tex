%%% LANDSCAPE BEGINS =======================================
%% \pagebreak
%% \clearpage
%%\begin{flushleft}
%%\begin{landscape}
	
% =========================================================
\subsection{App4-Summary main() function C-Code listing for RTAI}

% ==========================================================
\lstset{basicstyle=\footnotesize, numberstyle=\tiny\color{blue}, frame=single, numbers=left, firstnumber=1, stepnumber=1, numbersep=1pt, xleftmargin=2.0em, framexleftmargin=1.5em, xrightmargin=0.0em, breaklines=true, breakatwhitespace=false, breakindent=5pt, prebreak=\space, postbreak=\space }
% ==========================================================

\begin{lstlisting}[caption={App4-Summary main() function C-Code listing for RTAI}, label=App4-Summary main() function C-Code listing for RTAI]

// File: rtai-CNC14.c
// Date: May 05, 2018 10:04
// Author: WRY
// =========================================================
int main(int argc, char* argv[]) {
// =========================================================
WRY00_Print_DateTime_Usec(); 
printf("Bismillah from WRY executed in main().\n\n");
	
// FOR KEYBOARD INTERRUPT TO STOP (e.g. <Ctrl-C>) EXECUTION
	signal(SIGTERM, WRYcleanup);
	signal(SIGINT,  WRYcleanup);
	signal(SIGKILL, WRYcleanup);
// INVOKE FUNCTIONS SEQUENTIALLY TO RUN RTAI
	system("./load-rtai-modules.sh"); // LOAD RTAI KERNEL MODULES
	WRY01_Create_and_Initialize_Real_Time_Task_LXRT();
	WRY02_Get_Address_and_Name_of_Real_Time_Task_LXRT();
	WRY31_Make_Hard_Real_Timer_Run(); 
	WRY32_Check_Selected_Hard_Real_Time_Run_Mode(); 
	WRY33_Execute_Selected_Hard_Real_Time_Run_Mode();
	WRY04_Allow_NonRoot_Users_For_Hard_Real_Time();
	WRY05_Lock_Memory_Swapping_For_Hard_Real_Time();
	WRY06_ParallelPort_Request_StartUp_and_Enable_IRQ(); 
	WRY07_ParallelPort_Check_RT_Task_Running_Mode();
	WRY08_Start_Run_CNC_Machine();
	WRY09_Stop_Run_CNC_Machine();
	system("./unload-rtai-modules.sh"); // UNLOAD RTAI KERNEL MODULES

WRY00_Print_DateTime_Usec(); 
printf("Alhamdulillah from WRY executed in main().\n"); 
return (0);
}
\end{lstlisting}

\pagebreak
%% =========================================================
\subsection{App4-Full C-Code listing for Real Time (RTAI)}
% ==========================================================
\lstset{basicstyle=\footnotesize, numberstyle=\tiny\color{blue}, frame=single, numbers=left, firstnumber=1, stepnumber=1, numbersep=1pt, xleftmargin=2.0em, framexleftmargin=1.5em, xrightmargin=0.0em, breaklines=true, breakatwhitespace=false, breakindent=5pt, prebreak=\space, postbreak=\space }
% ==========================================================
%% \begin{lstlisting}[caption={Putting listing codes in landscape format}, label=how-to-display-in-landscape]
\begin{lstlisting}[caption={App4-Full C-Code listing for Real Time (RTAI)}, label=App4-Full C-Code listing for Real Time (RTAI)]
// File: rtai-CNC14.c
// Date: May 05, 2018 10:04
// Author: WRY
// 
// =========================================================
// (1) The prefixes WRY and TASK were used extensively in
// order to differentiate our own defined variables and functions 
// from standard C/C++ and RTAI libraries defined variables and functions.
// 
// (2) We must either link our C/C++ program against liblxrt or we 
// have to configure RTAI with "CONFIG_RTAI_LXRT_STATIC_INLINE=y".
// during linux kernel compilation.
//
// =========================================================
// STANDARD C/C++ HEADER FILES USED
// ========================================================= 

	#include <sched.h>	// Scheduler
	#include <stdio.h>	// Standard Input/Output 
	#include <stdlib.h> 	// Standard Library
	#include <time.h>	// For local date-time with usec
	#include <signal.h> 	// Signal for user interrupts, <Ctrl>-C to stop 
	#include <fcntl.h>  	// For file control
	#include <math.h>	// Mathematical functions
	#include <unistd.h> 	// Low level constants, types, function declarations
	#include <errno.h>	// Printing C errors
	#include <string.h>	// Handling strlen IN gcode . h
	#include <curses.h>	// Handling getch () , wgetch ()
	#include <sys/ioctl.h>	// System Input/Output control
	#include <sys/io.h>	// System Input/Output
	#include <sys/mman.h>	// System Memory Manager
	#include <sys/time.h>	// For system local date-time with usec




// ========================================================
// REALTIME HEADER FILES (RTAI)
// ========================================================
	#include <rtai.h>		// RTAI configuration switches (User space and Kernel space)
	#include <rtai_lxrt.h>  	// RTAI User Space libraries

// =========================================================
// RTAI VER. 5.1 DOCUMENTATION REFERENCES
// =========================================================
/*
	For full RTAI Ver 5.1 LOCAL DOCUMENTATION
		file:///usr/realtime/share/doc/rtai-5.1/html/api/files.html
	For <rtai.h>
		file:///usr/realtime/share/doc/rtai-5.1/html/api/rtai_8h_source.html
	For <rtai_lxrt.h>
		file:///usr/realtime/share/doc/rtai-5.1/html/api/rtai__lxrt_8h.html
	For <rtai_sem.h>
		file:///usr/realtime/share/doc/rtai-5.1/html/api/rtai__sem_8h.html
	For <rtai_sched.h>
		file:///usr/realtime/share/doc/rtai-5.1/html/api/rtai__sched_8h_source.html
	For <rtai_usi.h>
		file:///usr/realtime/share/doc/rtai-5.1/html/api/rtai__usi_8h_source.html
*/
// =========================================================
// EXAMPLE USAGE OF RTAI FUNCTIONS (INTERRUPT-BASED)
// ========================================================= 
/*	rt_set_oneshot_mode();
	rt_set_periodic_mode();
	start_rt_timer();
	rt_request_irq_task(PARPORT_IRQ, WRY_RT_task, RT_IRQ_TASK, 1)
	rt_task_init(); 
	rt_task_make_periodic(); 
	rt_make_hard_real_time();
	rt_make_soft_real_time();
	rt_allow_nonroot_hrt();
	rt_task_delete(WRY_RT_task);
	stop_rt_timer();
*/

// ========================================================-
// RTAI DEFINED FUNCTIONS
// =========================================================
/*
	static void * 		rt_get_adr (unsigned long name);
		Get an object address by its name. 
		Returns the address associated to name on success, 0 on failure 
	
	static unsigned long 	rt_get_name (void *adr);
		Get an object name by its address. 
		Returns the address associated to name on success, 0 on failure 
	
	static RT_TASK * 	rt_task_init (unsigned long name, int priority, int stack_size, int max_msg_size);
		Create an RTAI task extension for a Linux process/task in user space.
	
	static void 		rt_make_soft_real_time (void);
		Return a hard real time Linux process, or pthread to the standard Linux behavior.
	
	static void 		rt_make_hard_real_time (void);
		Give a Linux process, or pthread, hard real time execution capabilities allowing full kernel preemption
	
	static void 	rt_allow_nonroot_hrt (void);
		Allows a non root user to use the Linux POSIX soft real time process management and memory lock functions, and allows it to do any input-output operation from user space.
*/
// ========================================================
// GLOBAL DEFINITIONS
// ========================================================
	#define PERIOD		500000		// nanoseconds
	#define TICK_TIME	1000000		// nanoseconds
	// #define CPUMAP 	0xF  		// 16-cpus
	#define CPUMAP		0x4		// 4-cpus
	
	// BUILT_IN PARALLEL PORT ON MOTHERBOARD
	// #define PARPORT_IRQ	7
	// #define BASEPORT	0x378
	
	// HARDWARE 1 = parport0: PC-style at 0xd010 (0xd000), irq 18
	#define BASEPORT	0xd010
	#define PARPORT_IRQ	18

	// HARDWARE 2 = parport1: PC-style at 0xe100, irq 17
	// #define BASEPORT	0xe100
	// #define PARPORT_IRQ	17

	static volatile int ovr, intcnt, retval, maxcnt;

// =========================================================
// HARDWARE DEVICES PARALLEL PORT SEARCH
// =========================================================
/* EXECUTE SYSTEM FUNCTION
root@dell-ub1604-64b:/home/wruslan# dmesg | grep parp

KERNEL MESSAGE - FOUND PARALLEL PORT NO. 1
	[    8.989048] parport0: PC-style at 0xd010 (0xd000), irq 18, using FIFO [PCSPP,TRISTATE,COMPAT,EPP,ECP]
	[    9.089563] lp0: using parport0 (interrupt-driven).
	
KERNEL MESSAGE - FOUND PARALLEL PORT NO. 2
	[   11.972056] parport1: PC-style at 0xe100, irq 17 [PCSPP,TRISTATE]
	[   12.073309] lp1: using parport1 (interrupt-driven).

root@dell-ub1604-64b:/home/wruslan# lspci -v

LIST PCI DEVICE - FOUND PARALLEL PORT NO. 1
	03:00.0 Parallel controller: Oxford Semiconductor Ltd Device c110 (prog-if 02 [ECP])
	Subsystem: Oxford Semiconductor Ltd Device c110
	Flags: bus master, fast devsel, latency 0, IRQ 18
	I/O ports at d010 [size=8]
	I/O ports at d000 [size=4]
	Capabilities: [40] Power Management version 3
	Capabilities: [50] MSI: Enable- Count=1/1 Maskable- 64bit+
	Capabilities: [70] Express Legacy Endpoint, MSI 00
	Capabilities: [100] Device Serial Number 00-30-e0-11-11-00-01-10
	Capabilities: [110] Power Budgeting <?>
	Kernel driver in use: parport_pc
	Kernel modules: parport_pc

	
LIST PCI DEVICE - FOUND PARALLEL PORT NO. 2
	02:00.0 Serial controller: Device 1c00:3250 (rev 10) (prog-if 05 [16850])
	Subsystem: Device 1c00:3250
	Flags: fast devsel, IRQ 17
	I/O ports at e000 [size=256]
	Memory at f0100000 (32-bit, prefetchable) [size=32K]
	I/O ports at e100 [size=4]
	Expansion ROM at f7c00000 [disabled] [size=32K]
	Capabilities: [60] Power Management version 3
	Capabilities: [68] MSI: Enable- Count=1/32 Maskable+ 64bit+
	Capabilities: [80] Express Legacy Endpoint, MSI 00
	Capabilities: [100] Advanced Error Reporting
	Kernel driver in use: parport_serial
	Kernel modules: parport_serial

*/
// ========================================================
// REAL TIME OBJECTS
// ========================================================
	static RT_TASK* 	WRYtask;
	static RT_TASK*     	WRY_RT_task;
	struct sched_param 	WRYsched;
	struct timeval 		WRYstart, WRYfinish;
	unsigned long 		WRY_RT_task_name; 

// ========================================================
// TIMING FOR GENERAL CNC EXECUTION
// ========================================================
	static RTIME 		CNCunixtime1, CNCunixtime2;
	static RTIME		CNCstart_time_ns, CNCcurrent_time_ns;
	static RTIME		CNCend_time_ns, CNCrun_duration_ns;
	static RTIME 		CNCtimer_period_ns; // timer period, in nanoseconds
	RTIME 			CNCtimer_period_count; //actual timer period, in counts




// ========================================================
// TIMING FOR INDIVIDUAL TASK EXECUTION
// ========================================================
	struct timeval 		TASKstart, TASKfinish;
	long int		TASKusec_duration;
	double			TASKsec_duration;

// ========================================================
// GLOBAL VARIABLES (With KEYBOARD Interrupt)
// ========================================================
	int 			WRYpriority     = 0; // Highest
	int 			WRYstack_size   = 0; // Using default (512)
	int 			WRYmax_msg_size = 0; // Using default (256)
	int 			WRYpolicy 	= SCHED_FIFO;
	int 			WRYcpus_allowed	= CPUMAP;
	long int 		WRYusec_duration;
	double 			WRYsec_duration ;
	int			WRYsig;  // Keyboard interrupt signal <Ctrl>-C
	static volatile int 	ovr, intcnt, retval, maxcnt;

// =======================================================
// DEFINE REALTIME TASK STATUS AND SETTINGS
// =======================================================
// REFERNCE: Initially task status set to non-periodic (value non-zero)
// Later on in program we make task periodic (change value to 0)

	int 		WRY_RT_task_status = 1;  
	static RTIME 	WRYexpected;
	static RTIME 	WRYsampling_interval;	

// SELECT REAL TIME TIMER PERIOD
	static RTIME 	CNCtimer_period_ns = 1000*1000*1000; // means (1 sec) period

// SET REAL TIME RUN MODE (BOTH CANNOT BE TRUE)
	int 		WRY_set_periodic_mode = 1; // TRUE  = 1 
	int		WRY_set_one_shot_mode = 0; // FALSE = 0

// SELECTED RUN MODE VALUES (PERIODIC == 1, ONE-SHOT == 2)
	int 		WRY_selected_run_mode; 

// TASK PERIODIC STATUS VALUES (PERIODIC == 0, NON-PERIODIC != 0)
	int 		WRY_task_periodic_status;

// TASK ONE-SHOT STATUS VALUES (ONE-SHOT ???, NON-ONE-SHOT ???)
	int 		WRY_task_one_shot_status;




// =======================================================
// GLOBAL DATA STRUCTURES 
// =======================================================
// Used for WRY00_Print_DateTime_Usec(void) ONLY in order 
// to avoid repetitive declarations (reused) 

	time_t 		WRYtimer;
	char 		WRYbuffer[26];
	struct tm*	WRYtm_info;
	struct timeval	WRYtval_now;




// ========================================================
void WRY00_Print_DateTime_Usec(void) {
// ========================================================
// EXECUTIONS
	time(&WRYtimer);
	WRYtm_info = localtime(&WRYtimer);
	strftime(WRYbuffer, 26, "%Y-%m-%d %H:%M:%S", WRYtm_info);
	gettimeofday(&WRYtval_now, NULL);
	printf("%s", WRYbuffer);
	printf(".%09ld \t", (long int)WRYtval_now.tv_usec);
}




// =======================================================
void WRY01_Create_and_Initialize_Real_Time_Task_LXRT(void) {
// =======================================================
WRY00_Print_DateTime_Usec(); 
printf("STARTED  WRY01_Create_and_Initialize_Real_Time_Task-LXRT(void).\n");

// Create an RTAI task extension for a Linux process/task in user space.

	if (!(WRY_RT_task = rt_task_init(nam2num("CNC06"), WRYpriority, WRYstack_size, WRYmax_msg_size))) {
		WRY00_Print_DateTime_Usec();
		printf("ERROR : Cannot initialize real time task LXRT.\n");
		WRY00_Print_DateTime_Usec();
		printf("ERROR DESCRIPTION : %s\n", strerror(errno));
		exit(1);
	} else {
		WRY00_Print_DateTime_Usec();
		printf("SUCCESS: Completed real time task LXRT initialization.\n");
	}

WRY00_Print_DateTime_Usec(); 
printf("FINISHED WRY01_Create_and_Initialize_Real_Time_Task-LXRT(void).\n\n");
}

// =======================================================
void WRY02_Get_Address_and_Name_of_Real_Time_Task_LXRT(void) {
// =======================================================
WRY00_Print_DateTime_Usec(); 
printf("STARTED  WRY02_Get_Address_and_Name_of_Real_Time_Task_LXRT(void).\n");

	// GET RT_TASK NAME BY ADDRESS
	if (rt_get_name(WRY_RT_task) == 0) {
		WRY00_Print_DateTime_Usec();
		printf("ERROR : Cannot get name of real time task LXRT.\n");
		WRY00_Print_DateTime_Usec();
		printf("ERROR DESCRIPTION : %s\n", strerror(errno));
	//	exit(1);
	} else {
		WRY00_Print_DateTime_Usec();
		printf("SUCCESS: NAME of real time task LXRT = %p\n", (void *)WRY_RT_task);
		WRY_RT_task_name = (unsigned long)WRY_RT_task;
	}

WRY00_Print_DateTime_Usec(); 
printf("FINISHED WRY02_Get_Address_and_Name_of_Real_Time_Task_LXRT(void).\n\n");
}

// ========================================================
void WRY31_Make_Hard_Real_Timer_Run(void) {
// ========================================================
WRY00_Print_DateTime_Usec(); 
printf("STARTED  WRY31_Make_Hard_Real_Timer_Run(void).\n");

	// EXECUTE HARD REAL TIMER CLOCK
	rt_make_hard_real_time();
	
	// IF HARD REAL TIME ALREADY RUNNING, NOTIFY STATUS
	if (rt_is_hard_timer_running()) {
		WRY00_Print_DateTime_Usec();
		printf("SUCCESS: Hard real timer is already running.\n");
	} else {
		// IF NOT RUNNING, START WHILE LOOP
		while (!(rt_is_hard_timer_running())) {
	
			// EXECUTE HARD REAL TIMER CLOCK
			rt_make_hard_real_time();
			// IF HARD REAL TIMER IS NOT RUNNING, THEN RUN IT BY LOOPING.
			if (rt_is_hard_timer_running()) {
				WRY00_Print_DateTime_Usec();
				printf("SUCCESS: Hard real timer is NOW running.\n");
			} else {
				WRY00_Print_DateTime_Usec();
				printf("FAILED : Hard real timer is NOT YET running.\n");
				WRY00_Print_DateTime_Usec();
				printf("ERROR DESCRIPTION : %s\n", strerror(errno));
				// exit(1); // DISABLED TO KEEP ON TRYING
			} END if else    
		} // END while loop
	} // END if else

WRY00_Print_DateTime_Usec(); 
printf("FINISHED WRY31_Make_Hard_Real_Timer_Run(void).\n\n");
}

// ========================================================
void WRY32_Check_Selected_Hard_Real_Time_Run_Mode(void) {
// ========================================================
WRY00_Print_DateTime_Usec(); 
printf("STARTED  WRY32_Check_Selected_Hard_Real_Time_Run_Mode(void).\n");

	// BOTH RUN MODES SELECTED
	if ((WRY_set_periodic_mode == 1) && (WRY_set_one_shot_mode == 1)) {
		WRY00_Print_DateTime_Usec();
		printf("ERROR : Cannot select both periodic_mode and one-shot mode.\n");
		exit(1); 	
		} 
	// NO RUN MODE SELECTED
	if ((WRY_set_periodic_mode == 0) && (WRY_set_one_shot_mode == 0)) {
		WRY00_Print_DateTime_Usec();
		printf("ERROR : No hard realtime run mode selected.\n");
		exit(1); 	
		} 
	// PERIODIC MODE SELECTED
	if ((WRY_set_periodic_mode == 1) && (WRY_set_one_shot_mode == 0)) {
		WRY00_Print_DateTime_Usec();
		printf("SUCCESS: PERIODIC hard realtime run mode selected.\n");
		WRY_selected_run_mode = 1; 	
		} 
	// ONE-SHOT MODE SELECTED
	if ((WRY_set_periodic_mode == 0) && (WRY_set_one_shot_mode == 1)) {
		WRY00_Print_DateTime_Usec();
		printf("SUCCESS: ONE-SHOT hard realtime run mode selected.\n");
		WRY_selected_run_mode = 2;	
		}
WRY00_Print_DateTime_Usec(); 
printf("FINISHED WRY32_Check_Selected_Hard_Real_Time_Run_Mode(void).\n\n");
}

// ========================================================
void WRY33_Execute_Selected_Hard_Real_Time_Run_Mode(void) {
// ========================================================
WRY00_Print_DateTime_Usec(); 
printf("STARTED  WRY33_Execute_Selected_Hard_Real_Time_Run_Mode(void).\n");

	// FOR PERIODIC-MODE REAL TIME RUN
	if (WRY_selected_run_mode == 1) {
	
		WRY00_Print_DateTime_Usec();
		printf("SUCCESS: PERIODIC Set CNCtimer_period_ns \t= %lld \n", CNCtimer_period_ns);
	
		rt_set_periodic_mode();
		WRY00_Print_DateTime_Usec();
		printf("SUCCESS: PERIODIC Execute rt_set_periodic_mode().\n");

		CNCtimer_period_count=nano2count(CNCtimer_period_ns);
		WRY00_Print_DateTime_Usec();
		printf("SUCCESS: PERIODIC Execute nano2count(CNCtimer_period_ns).\n");

		WRY00_Print_DateTime_Usec();
		printf("SUCCESS: PERIODIC CNCtimer_period_count \t= %lld \n", CNCtimer_period_count);

		start_rt_timer(CNCtimer_period_count);
		WRY00_Print_DateTime_Usec();
		printf("SUCCESS: PERIODIC Execute start_rt_timer(CNCtimer_period_count).\n");

		rt_task_make_periodic(WRY_RT_task, WRYexpected, WRYsampling_interval);
		WRY00_Print_DateTime_Usec();
		printf("SUCCESS: PERIODIC Execute rt_task_make_periodic(WRY_RT_task, x, x).\n");

	} // END if PERIODIC-MODE

	// FOR ONE-SHOT MODE REAL TIME RUN
	if (WRY_selected_run_mode == 2) {
	
		rt_set_oneshot_mode();
		WRY00_Print_DateTime_Usec();
		printf("SUCCESS: ONE-SHOT Execute rt_set_oneshot_mode().\n");
		
		start_rt_timer(0);
		WRY00_Print_DateTime_Usec();
		printf("SUCCESS: ONE-SHOT Execute start_rt_timer(0).\n");
	
	} // END if ONE-SHOT MODE

WRY00_Print_DateTime_Usec(); 
printf("FINISHED WRY33_Execute_Selected_Hard_Real_Time_Run_Mode(void).\n\n");
}

// ========================================================
void WRY04_Allow_NonRoot_Users_For_Hard_Real_Time(void) {
// ========================================================
WRY00_Print_DateTime_Usec(); 
printf("STARTED  WRY04_Allow_NonRoot_Users_For_Hard_Real_Time(void).\n");

	// ALLOW NON-ROOT USERS TO HARD REAL TIME
	rt_allow_nonroot_hrt();
	WRY00_Print_DateTime_Usec();
	printf("SUCCESS: Execute Allow hard realtime for non-root users.\n");

WRY00_Print_DateTime_Usec(); 
printf("FINISHED WRY04_Allow_NonRoot_Users_For_Hard_Real_Time(void).\n\n");
}
// ========================================================
void WRY05_Lock_Memory_Swapping_For_Hard_Real_Time(void) {
// ========================================================
WRY00_Print_DateTime_Usec(); 
printf("STARTED  WRY05_Lock_Memory_Swapping_For_Hard_Real_Time(void).\n");

	// LOCK RAM MEMORY FROM MEMORY SWAPPING
	mlockall(MCL_CURRENT | MCL_FUTURE);
	WRY00_Print_DateTime_Usec();
	printf ("SUCCESS: Execute Lock memory and now no memory swapping for hard realtime.\n") ;

WRY00_Print_DateTime_Usec(); 
printf("FINISHED WRY05_Lock_Memory_Swapping_For_Hard_Real_Time(void).\n\n");
}

// ========================================================
void WRY06_ParallelPort_Request_StartUp_and_Enable_IRQ(void) {
// ========================================================
WRY00_Print_DateTime_Usec(); 
printf("STARTED  WRY06_ParallelPort_Request_StartUp_and_Enable_IRQ(void).\n");

	// The name says it, IRQs managed in user space
	// REQUEST IRQ FOR PARALLEL PORT TASK (WRY_RT_task)
		rt_request_irq_task(PARPORT_IRQ, WRY_RT_task, RT_IRQ_TASK, 1);
	
		if(!(rt_request_irq_task(PARPORT_IRQ, WRY_RT_task, RT_IRQ_TASK, 1))) {
			WRY00_Print_DateTime_Usec();
			printf("ERROR  : rt_request_irq_task(PARPORT_IRQ, WRY_RT_task, RT_IRQ_TASK, 1).\n");
			WRY00_Print_DateTime_Usec();
			printf("ERROR DESCRIPTION : %s\n", strerror(errno));
			exit(1); // Commented during testing only
		} else {
			WRY00_Print_DateTime_Usec();
			printf("SUCCESS: Display PARPORT_IRQ \t= %d\n", PARPORT_IRQ);
			
			WRY00_Print_DateTime_Usec();
			printf("SUCCESS: Display RT_IRQ_TASK \t= %d\n", RT_IRQ_TASK);
			
			WRY00_Print_DateTime_Usec();
			printf("SUCCESS: Display BASEPORT \t= 0x%02X\n", BASEPORT);
			
			WRY00_Print_DateTime_Usec();
			printf("SUCCESS: Display CPUMAP \t= 0x%02X\n", CPUMAP);
			
			WRY00_Print_DateTime_Usec();
			printf("SUCCESS: Display PERIOD \t= %d (ns)\n", PERIOD);
			
			WRY00_Print_DateTime_Usec();
			printf("SUCCESS: Display TICK_TIME \t= %d (ns)\n", TICK_TIME);
			
			WRY00_Print_DateTime_Usec();
			printf("SUCCESS: Execute rt_request_irq_task(PARPORT_IRQ, WRY_RT_task, RT_IRQ_TASK, 1).\n");
		} // END if else (rt_request)			


/* REFERENCE NOTES: ===============
		https://www.rtai.org/userfiles/documentation/magma/html/api/group__hal.html#ga74
		
		Often some of the above functions do equivalent things. Once more there is no way of doing it right except by knowing the hardware you are manipulating. Furthermore you must also remember that when you install a hard real time handler the related interrupt is usually disabled, unless you are overtaking one already owned by Linux which has been enabled by it. Recall that if have done it right, and interrupts do not show up, it is likely you have just to rt_enable_irq() your irq. 
		
	TO SOLVE PROBLEM FOR RTAI VER 5.1 = BOTH LINES BELOW ARE NOT NEEDED
		rt_startup_irq(PARPORT_IRQ);
		rt_enable_irq(PARPORT_IRQ);
============== END REFERENCE NOTES */

WRY00_Print_DateTime_Usec(); 
printf("FINISHED WRY06_ParallelPort_Request_StartUp_and_Enable_IRQ(void).\n\n");
}

// ========================================================
void  WRY07_ParallelPort_Check_RT_Task_Running_Mode(void) {
// ========================================================
WRY00_Print_DateTime_Usec(); 
printf("STARTED  WRY07_ParallelPort_Check_RT_Task_Running_Mode(void).\n");

	WRY00_Print_DateTime_Usec();
	printf("SUCCESS: Display WRY_selected_run_mode = %d\n", WRY_selected_run_mode);

	// FOR RUN PERIODIC MODE SELECTED
	if (WRY_selected_run_mode == 1) {
	
		WRY_task_periodic_status = rt_task_make_periodic(WRY_RT_task, WRYexpected, WRYsampling_interval);
	
		// IF TASK RUNNING IN PERIODIC MODE
		if (WRY_task_periodic_status == 0) {  
			WRY00_Print_DateTime_Usec();
			printf("SUCCESS: Display WRY_RT_task is ALREADY running in periodic mode.\n");
		} else {
			// LOOP WHILE NOT RUNNING IN PERIODIC MODE
			// MAKE RT_TASK RUN IN PERIODIC MODE
			while (WRY_task_periodic_status != 0) {  
	
				WRY_task_periodic_status = rt_task_make_periodic(WRY_RT_task, WRYexpected, WRYsampling_interval);
				WRY00_Print_DateTime_Usec();
				printf("SUCCESS: Execute rt_task_make_periodic(WRY_RT_task, WRYexpected, WRYsampling_interval).\n");

				if (WRY_task_periodic_status == 0) {  
					WRY00_Print_DateTime_Usec();
					printf("SUCCESS: Display WRY_RT_task is NOW running in periodic mode.\n");
				} // END if
				
			} // END while loop MAKE TASK PERIODIC
		} // END if..else TASK RUNNING
	} END if SELECTED MODE

/* REFERENCE NOTES: ================

	int rt_task_make_periodic(RT_TASK* task, RTIME start_time, RTIME period) 
		Make a task run periodically.
	
	rt_task_make_periodic mark the task task, previously created with rt_task_init(), as suitable for a periodic execution, with period period, when rt_task_wait_period() is called.
	The time of first execution is defined through start_time or start_delay. start_time is an absolute value measured in clock ticks. start_delay is relative to the current time and measured in nanoseconds.
	
	Parameters:
		task 	is a pointer to the task you want to make periodic.
		start_time 	is the absolute time to wait before the task start running, in clock ticks.
		period 	corresponds to the period of the task, in clock ticks.

	Return values:
		0 	on success. A negative value on failure as described below:
		EINVAL: task does not refer to a valid task.
================== END REFERENCE NOTES  */

WRY00_Print_DateTime_Usec(); 
printf("FINISHED WRY07_ParallelPort_Check_RT_Task_Running_Mode(void).\n\n");
}

// ========================================================
void WRY08_Start_Run_CNC_Machine(void) {
// ========================================================
WRY00_Print_DateTime_Usec(); 
printf("STARTED  WRY08_Start_Run_CNC_Machine(void).\n");

	// MAP START CNC RUNTIME IN NANO SECONDS
	CNCstart_time_ns = rt_get_time_ns();
	WRY00_Print_DateTime_Usec(); 
	printf("SUCCESS: Start CNC run at time \t= %lld (ns)\n", CNCstart_time_ns);

	// EXAMPLE EXECUTE OF CNC Signals File RUN HERE 
	int x; 
	
	// START FOR ... CONTROL LOOP THAT READS CNC SIGNALS FILE AND WRITE LINE-BY-LINE
	// THIS FOR LOOP IS A TEST FOR 10 LINES TO WRITE
	for (x = 0; x < 10; x++) {
		
		// START TIMER
		gettimeofday(&TASKstart, NULL); CNCunixtime1 = rt_get_time_ns();
		
		// EXECUTE TEST RUN IMPORTANT
		// NOTE: rt_sleep suspends execution of the caller task for  a time of delay internal count units. During this time  the CPU is used by other tasks. TIME: 1000000000 (ns) = 2893422000 (internal counts)

		// FOR COMPILATION TESTING ONLY
		// DIFFERENT HARDWARE DEVICES WILL WRITE DIFFERENTLY
		outb_p(25, BASEPORT);   // THIS IS WRITING FOR PARALLEL PORT ON LINUX

		WRY00_Print_DateTime_Usec(); 
		printf("TEST WRITING outb_p(25, BASEPORT) \n");

		// THIS SLEEP TIME IS FOR PUSLE FEEDRATE
		// rt_sleep(1000000000);
		rt_sleep(2893422000);

		WRY00_Print_DateTime_Usec(); 
		printf("run rt_sleep(2893422000) internal counts for x = %d ", x);

		// END TIMER
		CNCunixtime2 = rt_get_time_ns(); gettimeofday(&TASKfinish,NULL);
		printf("run duration = %lld (ns)\n", (CNCunixtime2 - CNCunixtime1));
		
	} // END FOR ... CONTROL LOOP (MEANS G-CODE PROCESSING ENDED)

	CNCend_time_ns = rt_get_time_ns();
	WRY00_Print_DateTime_Usec(); 
	printf("SUCCESS: End CNC run at time \t= %lld (ns)\n", CNCend_time_ns);
	
	CNCrun_duration_ns = (CNCend_time_ns - CNCstart_time_ns);
	WRY00_Print_DateTime_Usec(); 
	printf("SUCCESS: CNC total run duration\t= %lld (ns)\n", CNCrun_duration_ns);


/* REFERENCE NOTES ================= 

	CONSIDER RTAI SLEEP AND 
		rt_get_time_ns();   Nanoseconds timing in RTAI
		rt_sleep(1000000);  Equivalent to 1 second (rt_sleep is defined in micro seconds).
		rt_task_wait_period();
	
	REALTIME DRIVER FOR PARALLEL PORT OUTPUTS, MODIFIED FOR LINUX BY WRY
	REFER: Phase1D-CNC-RTOS - Report Page 161 of 204.
	
	outb_p(valout_pulse + on , BASEPORT);
	usleep(700);
	outb_p(valout_dir + on , BASEPORT);
	usleep(700);
	
	OUTB(2)              Linux Programmer's Manual                OUTB(2)	
	
	NAME
	
	outb, outw, outl, outsb, outsw, outsl, inb, inw, inl, insb, 
	insw, insl, outb_p, outw_p, outl_p, inb_p, inw_p, inl_p - port I/O

	SYNOPSIS
	#include <sys/io.h>

	unsigned char inb(unsigned short int port);
	unsigned char inb_p(unsigned short int port);
	unsigned short int inw(unsigned short int port);
	unsigned short int inw_p(unsigned short int port);
	unsigned int inl(unsigned short int port);
	unsigned int inl_p(unsigned short int port);

	void outb(unsigned char value, unsigned short int port);
	void outb_p(unsigned char value, unsigned short int port);
	void outw(unsigned short int value, unsigned short int port);
	void outw_p(unsigned short int value, unsigned short int port);
	void outl(unsigned int value, unsigned short int port);
	void outl_p(unsigned int value, unsigned short int port);

	void insb(unsigned short int port, void *addr,
	unsigned long int count);
	void insw(unsigned short int port, void *addr,
	unsigned long int count);
	void insl(unsigned short int port, void *addr,
	unsigned long int count);
	void outsb(unsigned short int port, const void *addr,
	unsigned long int count);
	void outsw(unsigned short int port, const void *addr,
	unsigned long int count);
	void outsl(unsigned short int port, const void *addr,
	unsigned long int count);

================== END REFERENCE NOTES */

WRY00_Print_DateTime_Usec(); 
printf("FINISHED WRY08_Start_Run_CNC_Machine(void).\n\n");
}

// =======================================================
void WRY09_Stop_Run_CNC_Machine(void) {
// ========================================================
WRY00_Print_DateTime_Usec(); 
printf("STARTED  WRY09_Stop_Run_CNC_Machine(void).\n");

	// RELEASE PARALLEL PORT 
	rt_release_irq_task(PARPORT_IRQ);
	WRY00_Print_DateTime_Usec(); 
	printf("SUCCESS: Execute rt_release_irq_task(PARPORT_IRQ).\n");
	
	// STOP REALTIME TIMER
	stop_rt_timer();
	WRY00_Print_DateTime_Usec(); 
	printf("SUCCESS: Execute stop_rt_timer().\n");
	
	// REVERT TO SOFT REALTIME
	rt_make_soft_real_time();
	WRY00_Print_DateTime_Usec(); 
	printf("SUCCESS: Execute rt_make_soft_real_time().\n");
	
	// DELETE REALTIME TASK
	rt_task_delete(WRY_RT_task);
	WRY00_Print_DateTime_Usec(); 
	printf("SUCCESS: Execute rt_task_delete(WRY_RT_task).\n");
	
	// NORMAL PROGRAM TERMINATION
	WRY00_Print_DateTime_Usec(); 
	printf("SUCCESS: Finished cleanup. Exiting program normally.\n");

WRY00_Print_DateTime_Usec(); 
printf("FINISHED WRY09_Stop_Run_CNC_Machine(void).\n\n");
}

// ========================================================
void WRYcleanup(int WRYsig) {
// ========================================================
printf("\n\n");
WRY00_Print_DateTime_Usec(); 
printf("STARTED  USER INTERRUPT RESPONSE TO STOP.\n");

	WRY00_Print_DateTime_Usec(); 
	printf("SUCCESS: Activate stop on User <Ctrl>-C Interrupt.\n");
	WRY00_Print_DateTime_Usec(); 
	printf("SUCCESS: Start cleanup sequence to exit program.\n");
	
	// RELEASE PARALLEL PORT 
	rt_release_irq_task(PARPORT_IRQ);
	WRY00_Print_DateTime_Usec(); 
	printf("SUCCESS: Execute rt_release_irq_task(PARPORT_IRQ).\n");
	
	// STOP REALTIME TIMER
	stop_rt_timer();
	WRY00_Print_DateTime_Usec(); 
	printf("SUCCESS: Execute stop_rt_timer().\n");
	
	// REVERT TO SOFT REALTIME
	rt_make_soft_real_time();
	WRY00_Print_DateTime_Usec(); 
	printf("SUCCESS: Execute rt_make_soft_real_time().\n");
	
	// ABNORMAL TERMINATION
	WRY00_Print_DateTime_Usec(); 
	printf("SUCCESS: Abnormal termination. Finished cleanup. Exiting program.\n");
	WRY00_Print_DateTime_Usec(); 
	printf("FINISHED USER INTERRUPT RESPONSE TO STOP.\n");
	
	// DELETE REALTIME TASK
	WRY00_Print_DateTime_Usec(); 
	printf("SUCCESS: Execute rt_task_delete(WRY_RT_task).\n\n");
	rt_task_delete(WRY_RT_task);

exit(1); // QUIT BASED ON KEYBOARD <Ctrl-C>
}




// ======================================================== 
int main(int argc, char* argv[]) {
// ========================================================
WRY00_Print_DateTime_Usec(); 
printf("Bismillah from WRY executed in main().\n\n");

// TO DO: TO INCLUDE AUTOMATIC RTAI MODULES load/unload

	// system("./load-rtai-modules.sh");
	
// FOR KEYBOARD INTERRUPT TO STOP (e.g. <Ctrl-C>)
	signal(SIGTERM, WRYcleanup);
	signal(SIGINT,  WRYcleanup);
	signal(SIGKILL, WRYcleanup);
	
// INVOKE FUNCTIONS SEQUENTIALLY TO RUN RTAI
	WRY01_Create_and_Initialize_Real_Time_Task_LXRT();
	WRY02_Get_Address_and_Name_of_Real_Time_Task_LXRT();
	WRY31_Make_Hard_Real_Timer_Run(); 
	WRY32_Check_Selected_Hard_Real_Time_Run_Mode(); 
	WRY33_Execute_Selected_Hard_Real_Time_Run_Mode();
	WRY04_Allow_NonRoot_Users_For_Hard_Real_Time();
	WRY05_Lock_Memory_Swapping_For_Hard_Real_Time();
	WRY06_ParallelPort_Request_StartUp_and_Enable_IRQ(); 
	WRY07_ParallelPort_Check_RT_Task_Running_Mode();
	WRY08_Start_Run_CNC_Machine();
	WRY09_Stop_Run_CNC_Machine();
	
	//system("./unload-rtai-modules.sh");

WRY00_Print_DateTime_Usec(); 
printf("Alhamdulillah from WRY executed in main().\n"); 
return (0);
}

// ======================================================
\end{lstlisting}

% =========================================================
% ============== BELOW IMPT TO MATCH \begin with \end ===== 
% \end{landscape}
% \end{flushleft}
% \clearpage
\pagebreak
% ==================== END OF PAGE ========================

% ==================== START OF PAGE ======================
% \pagebreak
% \clearpage
% \begin{flushleft}
%% \begin{landscape}
% =========================================================
\subsection{App4-Full execution C/C++ code for Real Time (RTAI)}
% ==========================================================
\lstset{basicstyle=\footnotesize, numberstyle=\tiny\color{blue}, frame=single, numbers=left, firstnumber=1, stepnumber=1, numbersep=1pt, xleftmargin=2.0em, framexleftmargin=1.5em, xrightmargin=0.0em, breaklines=true, breakatwhitespace=false, breakindent=5pt, prebreak=\space, postbreak=\space }
% ==========================================================
%% \begin{lstlisting}[caption={Putting listing codes in landscape format}, label=how-to-display-in-landscape]
\begin{lstlisting}[caption={App4-Full execution C/C++ code for Real Time (RTAI)}, label=App4-Full execution C/C++ code for Real Time (RTAI)]
/* COMPILATION AND EXECUTION RESULTS

(1) COMPILATION
wruslan@dell-ub1604-64b:~/ump-project2/test14$ gcc -I./ -I/usr/realtime/include -O2 -pipe -L/usr/realtime/lib -lpthread -llxrt -lm -o rtai-CNC14.x rtai-CNC14.c
wruslan@dell-ub1604-64b:~/ump-project2/test14$

(2) DISPLAY COMPILED EXECUTABLE 
wruslan@dell-ub1604-64b:~/ump-project2/test14$ ls -al
total 72
drwxrwxr-x  2 wruslan wruslan  4096 May  2 19:59 .
drwxrwxr-x 19 wruslan wruslan  4096 May  2 19:42 ..
-rwxrwxrwx  1 wruslan wruslan  4096 Apr 28 09:23 load-rtai-modules.sh
-rw-rw-r--  1 wruslan wruslan 31513 May  2 19:59 rtai-CNC14.c
-rwxrwxr-x  1 wruslan wruslan 24064 May  2 19:59 rtai-CNC14.x           <==== FOUND COMPILED EXECUTABLE
-rwxrwxrwx  1 wruslan wruslan  4063 Apr 28 09:28 unload-rtai-modules.sh
wruslan@dell-ub1604-64b:~/ump-project2/test14$

(3) EXECUTE COMPILED EXECUTABLE AS ROOT USER
wruslan@dell-ub1604-64b:~/ump-project2/test14$ sudo ./rtai-CNC14.x 
[sudo] password for wruslan: 
2018-05-02 19:59:42.000944348 	Bismillah from WRY executed in main().

2018-05-02 19:59:42.000945009 	SUCCESS: Completed real time task LXRT initialization.
2018-05-02 19:59:42.000945019 	SUCCESS: NAME of real time task LXRT = 0xffffa08f43997230
2018-05-02 19:59:42.000945038 	SUCCESS: Hard real timer is already running.
2018-05-02 19:59:42.000945045 	SUCCESS: PERIODIC hard realtime run mode selected.
2018-05-02 19:59:42.000945051 	SUCCESS: PERIODIC Set CNCtimer_period_ns 	= 1000000000 
2018-05-02 19:59:42.000945064 	SUCCESS: PERIODIC Execute rt_set_periodic_mode().
2018-05-02 19:59:42.000945076 	SUCCESS: PERIODIC Execute nano2count(CNCtimer_period_ns).
2018-05-02 19:59:42.000945083 	SUCCESS: PERIODIC CNCtimer_period_count 	= 3292519000 
2018-05-02 19:59:42.000945102 	SUCCESS: PERIODIC Execute start_rt_timer(CNCtimer_period_count).
2018-05-02 19:59:42.000945115 	SUCCESS: PERIODIC Execute rt_task_make_periodic(WRY_RT_task, x, x).
2018-05-02 19:59:42.000945128 	SUCCESS: Execute Allow hard realtime for non-root users.
2018-05-02 19:59:42.000945274 	SUCCESS: Execute Lock memory and now no memory swapping for hard realtime.
2018-05-02 19:59:42.000945281 	FINISHED WRY05_Lock_Memory_Swapping_For_Hard_Real_Time(void).

2018-05-02 19:59:42.000945288 	STARTED  WRY06_ParallelPort_Request_StartUp_and_Enable_IRQ(void).
2018-05-02 19:59:42.000945301 	SUCCESS: Display PARPORT_IRQ 	= 18
2018-05-02 19:59:42.000945308 	SUCCESS: Display RT_IRQ_TASK 	= 0
2018-05-02 19:59:42.000945315 	SUCCESS: Display BASEPORT 	= 0xD010
2018-05-02 19:59:42.000945322 	SUCCESS: Display CPUMAP 	= 0x04
2018-05-02 19:59:42.000945329 	SUCCESS: Display PERIOD 	= 500000 (ns)
2018-05-02 19:59:42.000945335 	SUCCESS: Display TICK_TIME 	= 1000000 (ns)
2018-05-02 19:59:42.000945342 	SUCCESS: Execute rt_request_irq_task(PARPORT_IRQ, WRY_RT_task, RT_IRQ_TASK, 1).
2018-05-02 19:59:42.000945348 	FINISHED WRY06_ParallelPort_Request_StartUp_and_Enable_IRQ(void).

2018-05-02 19:59:42.000945355 	STARTED  WRY07_ParallelPort_Check_RT_Task_Running_Mode(void).
2018-05-02 19:59:42.000945362 	SUCCESS: Display WRY_selected_run_mode = 1
2018-05-02 19:59:42.000945374 	SUCCESS: Display WRY_RT_task is ALREADY running in periodic mode.
2018-05-02 19:59:42.000945381 	FINISHED WRY07_ParallelPort_Check_RT_Task_Running_Mode(void).

2018-05-02 19:59:42.000945388 	STARTED  WRY08_Start_Run_CNC_Machine(void).
2018-05-02 19:59:42.000945400 	SUCCESS: Start CNC run at time 	= 1528453940458 (ns)
2018-05-02 19:59:42.000945418 	TEST WRITING outb_p(25, BASEPORT) 
2018-05-02 19:59:43.000824252 	run rt_sleep(2893422000) internal counts for x = 0 run duration = 878838671 (ns)
2018-05-02 19:59:43.000824285 	TEST WRITING outb_p(25, BASEPORT) 
2018-05-02 19:59:44.000703121 	run rt_sleep(2893422000) internal counts for x = 1 run duration = 878841400 (ns)
2018-05-02 19:59:44.000703155 	TEST WRITING outb_p(25, BASEPORT) 
2018-05-02 19:59:45.000581988 	run rt_sleep(2893422000) internal counts for x = 2 run duration = 878838807 (ns)
2018-05-02 19:59:45.000582021 	TEST WRITING outb_p(25, BASEPORT) 
2018-05-02 19:59:46.000460855 	run rt_sleep(2893422000) internal counts for x = 3 run duration = 878839578 (ns)
2018-05-02 19:59:46.000460888 	TEST WRITING outb_p(25, BASEPORT) 
2018-05-02 19:59:47.000339721 	run rt_sleep(2893422000) internal counts for x = 4 run duration = 878838160 (ns)
2018-05-02 19:59:47.000339753 	TEST WRITING outb_p(25, BASEPORT) 
2018-05-02 19:59:48.000218585 	run rt_sleep(2893422000) internal counts for x = 5 run duration = 878836793 (ns)
2018-05-02 19:59:48.000218617 	TEST WRITING outb_p(25, BASEPORT) 
2018-05-02 19:59:49.000097453 	run rt_sleep(2893422000) internal counts for x = 6 run duration = 878841787 (ns)
2018-05-02 19:59:49.000097487 	TEST WRITING outb_p(25, BASEPORT) 
2018-05-02 19:59:49.000976322 	run rt_sleep(2893422000) internal counts for x = 7 run duration = 878839913 (ns)
2018-05-02 19:59:49.000976355 	TEST WRITING outb_p(25, BASEPORT) 
2018-05-02 19:59:50.000855191 	run rt_sleep(2893422000) internal counts for x = 8 run duration = 878840543 (ns)
2018-05-02 19:59:50.000855223 	TEST WRITING outb_p(25, BASEPORT) 
2018-05-02 19:59:51.000734055 	run rt_sleep(2893422000) internal counts for x = 9 run duration = 878835850 (ns)
2018-05-02 19:59:51.000734082 	SUCCESS: End CNC run at time 	= 1537242485990 (ns)
2018-05-02 19:59:51.000734090 	SUCCESS: CNC total run duration	= 8788545532 (ns)
2018-05-02 19:59:51.000734097 	FINISHED WRY08_Start_Run_CNC_Machine(void).

2018-05-02 19:59:51.000734104 	STARTED  WRY09_Stop_Run_CNC_Machine(void).
2018-05-02 19:59:51.000734116 	SUCCESS: Execute rt_release_irq_task(PARPORT_IRQ).
2018-05-02 19:59:51.000734131 	SUCCESS: Execute stop_rt_timer().
2018-05-02 19:59:51.000734141 	SUCCESS: Execute rt_make_soft_real_time().
2018-05-02 19:59:51.000734150 	SUCCESS: Execute rt_task_delete(WRY_RT_task).
2018-05-02 19:59:51.000734156 	SUCCESS: Finished cleanup. Exiting program normally.
2018-05-02 19:59:51.000734162 	FINISHED WRY09_Stop_Run_CNC_Machine(void).

2018-05-02 19:59:51.000734171 	Alhamdulillah from WRY executed in main().
wruslan@dell-ub1604-64b:~/ump-project2/test14$ 
*/
\end{lstlisting}

% =========================================================
%\end{landscape}
%\end{flushleft}

