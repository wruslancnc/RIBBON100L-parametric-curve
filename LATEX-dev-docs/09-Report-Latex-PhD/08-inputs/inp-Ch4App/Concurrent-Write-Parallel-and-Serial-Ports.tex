% =========================================================
\subsection{App4-Concurrent Writes to Parallel and Serial Ports}\label{sec:App4-Write-Parallel-Serial-Ports}
% =========================================================

% ==========================================================
\lstset{basicstyle=\footnotesize, numberstyle=\tiny\color{blue}, frame=single, numbers=left, firstnumber=1, stepnumber=1, numbersep=1pt, xleftmargin=2.0em, framexleftmargin=1.5em, xrightmargin=0.0em, breaklines=true, breakatwhitespace=false, breakindent=5pt, prebreak=\space, postbreak=\space }
% ==========================================================

\begin{lstlisting}[caption={App4-Concurrent Writes to Parallel and Serial Ports}, label=App4-Concurrent Writes to Parallel and Serial Ports]

// File: Serial_Parallel_Write.c
// Date: Tue Nov 27 10:17:36 +08 2018
// Author: WRY
// ========================================================
/* Description

This C-program drives ASCII characters (8-bit integer numbers, 0 .. 255),
and generates TTL electrical pulses through two(2) USB ports on the personal
computer or notebook running Ubuntu Linux. The two USB attached devices 
basically extend the Linux computer system by: 

(1) USB-to-Serial PL2303 cable, making it a RS-232 serial port device.
(2) USB-to-Parallel PL2305 cable, making it a IEEE-1284 parallel port device.

Both cables are USB version 1.1 specification compliant.

In this C-code, as each character from a predefined character array, TheChar[],
is read, we write the character to the serial port and parallel port, 
sequentially, that is, one after another. 

Later on, depending on design, we may consider writing the character to the
two devices (serial and parallel ports) simultaneously, meaning, execution
parallel in time. 

The TTL electrical pulses are essentially the "8 bit" binary representation
of each character sent to the respective devices. We can observe these bits
using an oscilloscope. The more USB ports we extend on our computer (USB hosts), 
the more we can attach serial and parallel port devices. 

The PL2303 chip on the Prolific Technology USB-to-Serial cable device, 
is a full-duplex transmitter and receiver (TXD and RXD) for read and 
write serial communications.

The PL2305 chip on the USB-to-Parallel cable emulates a Bidirectional
printer device so can also be used for read and write communications. 

Later on, depending on design, we may consider both read and write 
communications using these devices. 

===========================================================
COMPUTING ENVIRONMENT
===========================================================
wruslan@HP-ELBook8470p-ub1604-64b:~$ date
Tue Nov 27 10:17:36 +08 2018

wruslan@HP-ELBook8470p-ub1604-64b:~$ uname -a
Linux HP-ELBook8470p-ub1604-64b 4.9.80rtai-5.1 #1 SMP Mon Apr 16 02:36:19 +08 2018 x86_64 x86_64 x86_64 GNU/Linux

wruslan@HP-ELBook8470p-ub1604-64b:~$ lsb_release -a
LSB Version:	core-9.20160110ubuntu0.2-amd64:core-9.20160110ubuntu0.2-noarch:security-9.20160110ubuntu0.2-amd64:security-9.20160110ubuntu0.2-noarch
Distributor ID:	Ubuntu
Description:	Ubuntu 16.04.5 LTS
Release:	16.04
Codename:	xenial

============================================================
SERIAL PORT SOFTWARE DRIVER (Cable USB-to-Serial PL2303)
============================================================
wruslan@HP-ELBook8470p-ub1604-64b:~$ ls -al /dev/ | grep ttyU
crw-rw----   1 root    dialout   188,   0 Nov 27 10:19 ttyUSB0
wruslan@HP-ELBook8470p-ub1604-64b:~$ 

wruslan@HP-ELBook8470p-ub1604-64b:~$ lsmod | grep pl
pl2303                 20480  0
usbserial              53248  1 pl2303

wruslan@HP-ELBook8470p-ub1604-64b:~$ dmesg
....
[ 6428.591444] usb 1-1.6: new full-speed USB device number 24 using ehci-pci
[ 6428.702458] usb 1-1.6: New USB device found, idVendor=067b, idProduct=2303
[ 6428.702464] usb 1-1.6: New USB device strings: Mfr=1, Product=2, SerialNumber=0
[ 6428.702467] usb 1-1.6: Product: USB-Serial Controller
[ 6428.702470] usb 1-1.6: Manufacturer: Prolific Technology Inc.
[ 6428.703108] pl2303 1-1.6:1.0: pl2303 converter detected
[ 6428.705060] usb 1-1.6: pl2303 converter now attached to ttyUSB0
....
============================================================
PARALLEL PORT SOFTWARE DRIVER (Cable USB-to-Parallel PL2305)
============================================================
wruslan@HP-ELBook8470p-ub1604-64b:~$ ls -al /dev/usb/lp0
crw-rw---- 1 root lp 180, 0 Nov 26 22:43 /dev/usb/lp0

wruslan@HP-ELBook8470p-ub1604-64b:~$ lsmod | grep usbl
usblp                  20480  0

wruslan@HP-ELBook8470p-ub1604-64b:~$ dmesg
....
[ 6451.118420] usb 1-1.2: new full-speed USB device number 26 using ehci-pci
[ 6451.228661] usb 1-1.2: New USB device found, idVendor=067b, idProduct=2305
[ 6451.228667] usb 1-1.2: New USB device strings: Mfr=1, Product=2, SerialNumber=0
[ 6451.228670] usb 1-1.2: Product: IEEE-1284 Controller
[ 6451.228673] usb 1-1.2: Manufacturer: Prolific Technology Inc.
[ 6451.235287] usblp 1-1.2:1.0: usblp0: USB Bidirectional printer dev 26 if 0 alt 1 proto 2 vid 0x067B pid 0x2305
....

// REFERENCES: LINUX SERIAL PORT:
// https://stackoverflow.com/questions/13474923/read-dev-ttyusb0-with-a-c-program-on-linux
// https://stackoverflow.com/questions/10710083/serial-programming-on-linux-in-c
// https://www.codeproject.com/Questions/718340/C-program-to-Linux-Serial-port-read-write
// http://xanthium.in/Serial-Port-Programming-on-Linux
// https://github.com/xanthium-enterprises/Serial-Port-Programming-on-Linux/blob/master/USB2SERIAL_Write/Transmitter%20(PC%20Side)/SerialPort_write.c

*/
// ===========================================
// C-PROGRAM HEADER FILES
#include <stdio.h>
#include<string.h>
#include <unistd.h>
#include <stdlib.h>
#include <fcntl.h>
#include <sys/fcntl.h>
#include <termios.h>

// DEVICE: For-Cable-USB-to-Serial-PL2303.
#define DEVICE_PORT_SER	"/dev/ttyUSB0"

// DEVICE: For-Cable-USB-to-Parallel-PL2305.
#define DEVICE_PORT_PAR	"/dev/usb/lp0"

// PROTOTYPE FUNCTION DEFINITIONS
void convert_integer_to_binary8 (int input_int, char *output_bin);
void display_SERIAL_write_to_screen(int intIndex, int intWrite) ;
void display_PARALLEL_write_to_screen(int intIndex, int intWrite); 

// GLOBAL VARIABLE DECLARATIONS
// FOLLOW THE ORDER OF USING int main(int argc, char *argv[])
char *TheChar = "Bismillah Alhamdulillah !";

// BINARY REPRESENTATION FOR PRINTING (DISPLAY) TO SCREEN
char bin8_output[33] = {0};	// LAST CHAR IS NULL "\0"

// ========================================================
void convert_integer_to_binary8(int input_int, char *output_bin8) {
// ========================================================
	unsigned int mask8;
	mask8  = 0b10000000; 
	// printf("INTEGER INPUT = %d \n", input_int);

	int bitposition = 0;    
	while (mask8)  {         		// Loop until MASK is empty

		bitposition++;
		if (input_int & mask8) {     	// Bitwise AND => test the masked bit
			*output_bin8 = '1';     // if true, binary value 1 is appended to output array

		} else {
			*output_bin8 = '0';     // if false, binary value 0 is appended to output array
		}

		output_bin8++;                	// next character
		mask8 >>= 1;                 	// shift the mask variable 1 bit to the right
	}
	*output_bin8 = 0;                 	// add the trailing null 
}

// ========================================================
void display_SERIAL_write_to_screen(int intIndex, int intWrite) {
// ========================================================

	if (intWrite >= 1 && intWrite <=255) {
		convert_integer_to_binary8(intWrite, bin8_output);	
		printf("SERIAL   write TheChar[%d] INT= %d \tHEX= 0x%x \tBIN= %s \tASCII= %c \n", intIndex, intWrite, intWrite, bin8_output, intWrite);
	} else {
		printf("ERROR: Integer to write is outside of range (0..255) \n"); 	
	}
	// usleep(100000); // 0.1 sec delay (usleep = microsecond)
}
// ========================================================
void display_PARALLEL_write_to_screen(int intIndex, int intWrite) {
// ========================================================

	if (intWrite >= 1 && intWrite <=255) {
		convert_integer_to_binary8(intWrite, bin8_output);	
		printf("PARALLEL write TheChar[%d] INT= %d \tHEX= 0x%x \tBIN= %s \tASCII= %c \n", intIndex, intWrite, intWrite, bin8_output, intWrite);
	} else {
		printf("ERROR: Integer to write is outside of range (0..255) \n"); 	
	}
	// usleep(100000); // 0.1 sec delay (usleep = microsecond)
}
// ========================================================
int main(int argc, char *argv[]) {
// ========================================================
printf("Bismillah from WRY executed in main().\n\n");

	// OPEN SERIAL PORT
	// ==========================
	// Set file descriptor device to write: Cable USB-to-Serial PL2303
	int intFD_SER = open(DEVICE_PORT_SER, O_WRONLY | O_NOCTTY | O_NDELAY); 
	if(intFD_SER < 0) {
		perror(DEVICE_PORT_SER);
		printf("FAILED: SERIAL PORT. Cannot open file descriptor %s in main().\n", DEVICE_PORT_SER);
		printf("Value int file descriptor (fd_SER = %d) \n\n", intFD_SER);
		// exit(1);
	} else {
		printf("SUCCESS: SERIAL PORT. Opened file descriptor %s in main().\n", DEVICE_PORT_SER);
		printf("Value int file descriptor (fd_SER = %d) \n\n", intFD_SER);
	}
	
	// OPEN PARALLEL PORT
	// ==========================
	// Set file descriptor device to write: Cable USB-to-Parallel PL2305
	int intFD_PAR = open(DEVICE_PORT_PAR, O_WRONLY | O_NOCTTY | O_NDELAY); 
	if(intFD_PAR < 0) {
		perror(DEVICE_PORT_PAR);
		printf("FAILED: PARALLEL PORT. Cannot open file descriptor %s in main().\n", DEVICE_PORT_PAR);
		printf("Value int file descriptor (fd_PAR = %d) \n\n", intFD_PAR);
		// exit(1);
	} else {
		printf("SUCCESS: PARALLEL PORT. Opened file descriptor %s in main().\n", DEVICE_PORT_PAR);
		printf("Value int file descriptor (fd_PAR = %d) \n\n", intFD_PAR);
	} 
	
	// PERFORM WRITE LOOP FOR ELECTRICAL PULSE OUTPUTS
	int intIndex;
	int intToWrite = 0;
	
	for (intIndex = 0; intIndex < strlen(TheChar); intIndex++) {
	
		// GRAB INTEGER TO WRITE (8-BITS) TO DEVICE
		intToWrite = TheChar[intIndex]; 
	
		// (1) EXECUTE DEVICE write() ELECTRICAL PULSES TO SERIAL PORT 
		if (intFD_SER != -1) {
			write(intFD_SER, (const void *)TheChar[intIndex], 1); 
			display_SERIAL_write_to_screen(intIndex, intToWrite);
		}

		// (2) EXECUTE DEVICE write() ELECTRICAL PULSES TO PARALLEL PORT
		if (intFD_PAR != -1) {
			write(intFD_PAR, (const void *)TheChar[intIndex], 1); 
			display_PARALLEL_write_to_screen(intIndex, intToWrite);
		}

	} // END FOR..LOOP

	// CLOSE SERIAL AND PARALLEL PORTS
	if (intFD_SER >= 0) { close(intFD_SER); }
	if (intFD_PAR >= 0) { close(intFD_PAR); }

printf("\nAlhamdulillah from WRY executed in main().\n\n");
return(0);
}
// ========================================================
/* COMPILATION 

wruslan@HP:~$ gcc -w -o Serial_Parallel_Write.x Serial_Parallel_Write.c 

-rw-rw-r-- 1 wruslan wruslan 14183 Nov 27 14:25 Serial_Parallel_Write.c
-rwxrwxr-x 1 wruslan wruslan 13264 Nov 27 14:27 Serial_Parallel_Write.x

EXECUTION
=============================================================
wruslan@HP:~$ sudo ./Serial_Parallel_Write.x 

[sudo] password for wruslan: 
Bismillah from WRY executed in main().

SUCCESS: SERIAL PORT. Opened file descriptor /dev/ttyUSB0 in main().
Value int file descriptor (fd_SER = 3) 

SUCCESS: PARALLEL PORT. Opened file descriptor /dev/usb/lp0 in main().
Value int file descriptor (fd_PAR = 4) 

SERIAL   write TheChar[0] INT= 66 	HEX= 0x42 	BIN= 01000010 	ASCII= B 
PARALLEL write TheChar[0] INT= 66 	HEX= 0x42 	BIN= 01000010 	ASCII= B 
SERIAL   write TheChar[1] INT= 105 	HEX= 0x69 	BIN= 01101001 	ASCII= i 
PARALLEL write TheChar[1] INT= 105 	HEX= 0x69 	BIN= 01101001 	ASCII= i 
SERIAL   write TheChar[2] INT= 115 	HEX= 0x73 	BIN= 01110011 	ASCII= s 
PARALLEL write TheChar[2] INT= 115 	HEX= 0x73 	BIN= 01110011 	ASCII= s 
SERIAL   write TheChar[3] INT= 109 	HEX= 0x6d 	BIN= 01101101 	ASCII= m 
PARALLEL write TheChar[3] INT= 109 	HEX= 0x6d 	BIN= 01101101 	ASCII= m 
SERIAL   write TheChar[4] INT= 105 	HEX= 0x69 	BIN= 01101001 	ASCII= i 
PARALLEL write TheChar[4] INT= 105 	HEX= 0x69 	BIN= 01101001 	ASCII= i 
SERIAL   write TheChar[5] INT= 108 	HEX= 0x6c 	BIN= 01101100 	ASCII= l 
PARALLEL write TheChar[5] INT= 108 	HEX= 0x6c 	BIN= 01101100 	ASCII= l 
SERIAL   write TheChar[6] INT= 108 	HEX= 0x6c 	BIN= 01101100 	ASCII= l 
PARALLEL write TheChar[6] INT= 108 	HEX= 0x6c 	BIN= 01101100 	ASCII= l 
SERIAL   write TheChar[7] INT= 97 	HEX= 0x61 	BIN= 01100001 	ASCII= a 
PARALLEL write TheChar[7] INT= 97 	HEX= 0x61 	BIN= 01100001 	ASCII= a 
SERIAL   write TheChar[8] INT= 104 	HEX= 0x68 	BIN= 01101000 	ASCII= h 
PARALLEL write TheChar[8] INT= 104 	HEX= 0x68 	BIN= 01101000 	ASCII= h 
SERIAL   write TheChar[9] INT= 32 	HEX= 0x20 	BIN= 00100000 	ASCII=   
PARALLEL write TheChar[9] INT= 32 	HEX= 0x20 	BIN= 00100000 	ASCII=   
SERIAL   write TheChar[10] INT= 65 	HEX= 0x41 	BIN= 01000001 	ASCII= A 
PARALLEL write TheChar[10] INT= 65 	HEX= 0x41 	BIN= 01000001 	ASCII= A 
SERIAL   write TheChar[11] INT= 108 	HEX= 0x6c 	BIN= 01101100 	ASCII= l 
PARALLEL write TheChar[11] INT= 108 	HEX= 0x6c 	BIN= 01101100 	ASCII= l 
SERIAL   write TheChar[12] INT= 104 	HEX= 0x68 	BIN= 01101000 	ASCII= h 
PARALLEL write TheChar[12] INT= 104 	HEX= 0x68 	BIN= 01101000 	ASCII= h 
SERIAL   write TheChar[13] INT= 97 	HEX= 0x61 	BIN= 01100001 	ASCII= a 
PARALLEL write TheChar[13] INT= 97 	HEX= 0x61 	BIN= 01100001 	ASCII= a 
SERIAL   write TheChar[14] INT= 109 	HEX= 0x6d 	BIN= 01101101 	ASCII= m 
PARALLEL write TheChar[14] INT= 109 	HEX= 0x6d 	BIN= 01101101 	ASCII= m 
SERIAL   write TheChar[15] INT= 100 	HEX= 0x64 	BIN= 01100100 	ASCII= d 
PARALLEL write TheChar[15] INT= 100 	HEX= 0x64 	BIN= 01100100 	ASCII= d 
SERIAL   write TheChar[16] INT= 117 	HEX= 0x75 	BIN= 01110101 	ASCII= u 
PARALLEL write TheChar[16] INT= 117 	HEX= 0x75 	BIN= 01110101 	ASCII= u 
SERIAL   write TheChar[17] INT= 108 	HEX= 0x6c 	BIN= 01101100 	ASCII= l 
PARALLEL write TheChar[17] INT= 108 	HEX= 0x6c 	BIN= 01101100 	ASCII= l 
SERIAL   write TheChar[18] INT= 105 	HEX= 0x69 	BIN= 01101001 	ASCII= i 
PARALLEL write TheChar[18] INT= 105 	HEX= 0x69 	BIN= 01101001 	ASCII= i 
SERIAL   write TheChar[19] INT= 108 	HEX= 0x6c 	BIN= 01101100 	ASCII= l 
PARALLEL write TheChar[19] INT= 108 	HEX= 0x6c 	BIN= 01101100 	ASCII= l 
SERIAL   write TheChar[20] INT= 108 	HEX= 0x6c 	BIN= 01101100 	ASCII= l 
PARALLEL write TheChar[20] INT= 108 	HEX= 0x6c 	BIN= 01101100 	ASCII= l 
SERIAL   write TheChar[21] INT= 97 	HEX= 0x61 	BIN= 01100001 	ASCII= a 
PARALLEL write TheChar[21] INT= 97 	HEX= 0x61 	BIN= 01100001 	ASCII= a 
SERIAL   write TheChar[22] INT= 104 	HEX= 0x68 	BIN= 01101000 	ASCII= h 
PARALLEL write TheChar[22] INT= 104 	HEX= 0x68 	BIN= 01101000 	ASCII= h 
SERIAL   write TheChar[23] INT= 32 	HEX= 0x20 	BIN= 00100000 	ASCII=   
PARALLEL write TheChar[23] INT= 32 	HEX= 0x20 	BIN= 00100000 	ASCII=   
SERIAL   write TheChar[24] INT= 33 	HEX= 0x21 	BIN= 00100001 	ASCII= ! 
PARALLEL write TheChar[24] INT= 33 	HEX= 0x21 	BIN= 00100001 	ASCII= ! 

Alhamdulillah from WRY executed in main().

wruslan@HP:~$ 
*/
\end{lstlisting}
%% =========== END LISTING =================================
