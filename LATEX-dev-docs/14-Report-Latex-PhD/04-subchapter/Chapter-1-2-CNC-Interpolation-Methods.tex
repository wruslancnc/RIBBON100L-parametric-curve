\justifying

%%% Nothing EXPLANATION OF PROCESS FLOW
%% CREATE THE CNC OVERALL FLOW CHART
%\input{./diagrams/dia-Ch1/Flow-Chart-for-CNC-System.tex}

% ==========================================
\subsection{CNC Control Loop design}

The actual control loop design varies between different kinds of CNC machines. However, they share the same concepts and principles, that is, to run the system correctly obeying their designed parameters and limits. 
\vspace*{1\baselineskip}

The following are some control methods: open-loop control, closed-loop control, position control, velocity control, torque control, iterative control, adaptive control, compensation control, predictive control, artificial intelligence control, optimal control, security control, safety control, and so on. These control methods or strategies will be addressed in the section on literature review. 
% \vspace*{1\baselineskip}

\subsubsection{Control objectives}

In general, the first objective of control is to ensure that the system performs its specified function. The second control objective is to ensure that any variations in performance must stay within its prescribed and designed limits. A successful control design must satisfy these two objectives. 
\vspace*{1\baselineskip}

A control system consisting of interconnected components is designed to achieve a desired purpose. The standard software, engineering practice today, is to build systems with modular component architecture. In modular designs, new components can be added and existing components can be modified or removed without affecting the overall behaviour of the system. 
% \vspace*{1\baselineskip}

\subsubsection{Control loop operations}

For example, the CNC software control loop operates by invoking and executing separate functional components according to some control algorithm. Depending on the design of the control algorithm, the invoked components may run sequentially,  concurrently or in a parallel manner. (parallelly is acceptable in old English usage ... ha ha ha). 
\vspace*{1\baselineskip}

The control loop is considered the sole director that is tasked to orchestrate the interaction and interfacing among the different component functions within the system, and with the external environment. Thus, the construction of the control loop algorithm is the most important software component in the system. It is called a loop because it is supposed to run in-resident, stay continuously active, waiting to service any command request as it comes, and act to those events accordingly. The control loop is a service or daemon type of running program. If the control loop program dies, the entire system dies.
% \vspace*{1\baselineskip}

% ===========================================
% \pagebreak
\subsubsection{Control concepts}    

Control algorithms are designed using to the following concepts (PRMCC).

\begin{enumerate}
	\item a plan - the step-by-step sequence of actions to follow
	
	\item a review - the action of monitoring some parameters, act accordingly if the values are out of range   
	
	\item a measure - the action of taking measurement of some parameters, used for progress checking and decision making
	
	\item a coordination - the action of timely communicating with other components inside and outside the system 
	
	\item a control action - the decision on what action to take given the current state of the system.
\end{enumerate}

\subsubsection{Programming principles in control}

In software design for control algorithms, the following are a few programming principles that apply:
\begin{enumerate}
	\item process-driven programming - this is a step-by-step execution of functions, pre-planned and laid out to achieve some defined objectives 
	
	\item interrupt-driven programming - this is a request or notification event to the control director that an event occurred which required action 
	
	\item target-driven programming - this is a specification contract that the software must achieve the target by whatever means necessary
	
	\item optimal-control programming - this is a specification contract such that software parameters stay within certain agreed limits by whatever means necessary
	
\end{enumerate}

\subsubsection{Issues in control loop algorithm}

For the CNC control loop algorithm, all of the above mentioned concepts and principles apply. The design of the algorithm is proprietary. No sensible manufacturer will reveal this secret. In this research, we will build our own CNC control loop algorithm.
\vspace*{1\baselineskip}

For the CNC control loop in particular, for process-control the control algorithm must handle machine services, like start and stop the motors, monitor tool locations, monitor path tracking errors, monitor velocity limits, etc, in order to achieve process targets and act accordingly.
\vspace*{1\baselineskip}

For safety-control, the control algorithm must monitor temperature limits, friction limits, heating limits, vibrations, limits, etc, in order to service interrupt events and act accordingly.   
\vspace*{1\baselineskip}

For optimal control, the control algorithm must act by invoking and executing appropriate functions accordingly in order to achieve the optimal value (target maximum or minimum) of some dependent variable based on current values of independent variables or parameters. 
\vspace*{1\baselineskip}

Essentially, the control actions in the CNC control loop program work in a similar manner to a human being who has to make simultaneous control decisions and acts accordingly when faced with many things to do at one time. The control sensors for a human being is akin to the monitored parameters in the CNC machine. 
\vspace*{1\baselineskip}

The CNC control loop simultaneous multiple objectives are, as examples: to minimize contour error in tool path tracking, to ensure tool motion smoothness, to ensure the machining job is completed within reasonable time, to ensure minimum vibrations, to ensure heating is within tolerance, and so on, but not to forget that the machining job must be executed safely.  
% \vspace*{1\baselineskip}

\begin{tcolorbox}[colback=green!15!white,colframe=red!75!black,title=Research consideration no. 1]
As a consequence, the design of the CNC control loop is not an easy task. It is one of the challenges that we will undertake in this research project. We will discuss more on this in the section on literature review.
\end{tcolorbox}

% ==============================================
% \pagebreak
\subsection{Commercial CNC Control Software}

Based on the best CNC control software survey in 2017 published on the internet, \cite{Warfield_2017}, the following are some results for large industrial CNC machines, and small and medium shop-hobbyist type CNC systems.
% \vspace*{1\baselineskip}

\subsubsection{High end CNC machines}

For large and commercial CNC machines, the CNC control software are built special-purpose for the respective machines. Some famous brand names are as follows:
\begin{enumerate}
	\item Fanuc, \cite{FANUC_CNC}.
	\item Haas, \cite{HAAS_CNC}.
	\item Mazak, \cite{MAZAK_CNC}.
	\item Siemens, \cite{SIEMENS_CNC}.
	\item Centroid, \cite{CENTROID_CNC}.
	\item Heidenhain, \cite{HEIDENHAIN_CNC}.
	\item Mitsubishi, \cite{MITSUBISHI_CNC}.
	\item Allen-Bradley \cite{ALLEN-BRADLEY_CNC}.
\end{enumerate}

\subsubsection{Low end CNC Machines}
For small and medium size commercial CNC machines, the CNC control software are similarly built special-purpose for the respective machines. Some famous brand names are as follows:
\begin{enumerate}
	\item Mach CNC, \cite{MACH_CNC}.
	\item PathPilot/LinuxCNC, \cite{PATHPILOT_CNC}.
	\item Probotix CNC, \cite{PROBOTIX_CNC}.
	\item Planet CNC, \cite{PLANET_CNC}.
	\item Eding CNC, \cite{EDING_CNC}
	\item UCCNC Motion Control. \cite{UCCNC_CNC}.
\end{enumerate}
\subsubsection{Survey results}

\begin{tcolorbox}[colback=green!15!white,colframe=red!75!black,title=Research consideration no. 1]
The survey concluded that FANUC and HAAS are way ahead of everyone else at the high end market, while Mach3 and PathPilot-LinuxCNC rule the low end market. In this proposed research, we will be dealing only with LinuxCNC because it is cost-free but PathPilot is not.
\end{tcolorbox}
%%% =================================

