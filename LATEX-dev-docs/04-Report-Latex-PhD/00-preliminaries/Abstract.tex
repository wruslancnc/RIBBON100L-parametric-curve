\justifying
\vspace*{4\baselineskip}
{\LARGE\bfseries 
	Abstract\\
} 

\vspace*{1\baselineskip}

\begin{tcolorbox}	
\textit{\textbf{TO REMOVE LATER}: The abstract is a brief summary of your Ph.D. Research Proposal, and should be no longer than 200 words. It starts by describing in a few words the knowledge domain where your research takes place and the key issues of that domain that offer opportunities for the scientific or technological innovations you intend to explore. Taking those key issues as a background, you then present briefly your research statement, your proposed research approach, the results you expect to achieve, and the anticipated implications of such results on the advancement of the knowledge domain.}
\vspace*{1\baselineskip}

\textit{\textbf{TO REMOVE LATER}: To keep your abstract concise and objective, imagine that you were looking for financial support from someone who is very busy. Suppose that you were to meet that person at an official reception and that she would be willing to listen to you for no more than two minutes. What you would say to that person, and the pleasant style you would adopt in those two demanding minutes, is what you should put in your abstract. The guidelines provided in this template are meant to be used creatively and not, by any means, as a cookbook recipe for the production of research proposals.}
\end{tcolorbox}

\vspace*{1\baselineskip}
\normalsize
\textbf{Keywords}
\vspace*{1\baselineskip}

CNC, interpolation, reference-pulse, look-ahead control, feedrate compensation, realtime, realtime processing, parallel computing, parallel algorithm. 
\vspace*{1\baselineskip}

\begin{tcolorbox}
\textit{\textbf{TO REMOVE LATER}: This section is an alphabetically ordered list of the more appropriate words or expressions (up to twelve) that you would introduce in a search engine to find a research proposal identical to yours. The successive keywords are separated by comas.}
\end{tcolorbox}

%% =======================================================