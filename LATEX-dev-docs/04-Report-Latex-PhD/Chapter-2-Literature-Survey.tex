\pagebreak
\titleformat {\chapter} {\normalfont\huge\bfseries\color{black}}   {\thechapter}{10pt}{\huge} 
\chapter {Literature Survey}

Semi-systematic review with a table of comparisons. In the appendix. Only summary here.

\section{Reviews - CNC State of the Art}

\cite{Review_Lee_2018} CNC Algorithms for Precision Machining: State of the Art Review\\
	
As geometry of machined parts becomes complex the demands for more precise and faster machining using advanced computerized numerical control (CNC) are increased. Especially, recently improved computing power of CNC enables the implementation of the complicated control algorithms. Consequently a variety of intelligent control algorithms have been studied and implemented in CNC. This paper reviews the recent progress of control technologies for precision machining using CNC in the area of interpolation, contour control and compensation. 

In terms of interpolation several corner blending methods and parametric curves are introduced and the characteristics of each method are discussed. Regarding contour control algorithms recently developed multi-axis contour control methods are reviewed.

\cite{Review_Lasemi_2010} Recent development in CNC machining of freeform surfaces: A state-of-the-art review\\	



\cite{Koren_1976} Interpolator for a CNC System	\\
\cite{Koren_1979} Design of Computer Control for Manufacturing Systems	\\


\cite{Review_Lasemi_2010} -  Freeform surfaces, also called sculptured surfaces, have been widely used in various engineering applications. Freeform surfaces are primarily manufactured by CNC machining, especially 5-axis CNC machining. Various methodologies and computer tools have been developed in the past to improve efficiency and quality of freeform surface machining. This paper aims at providing a state-of-the-art review on recent research development in CNC machining of freeform surfaces. This review primarily focuses on three aspects in freeform surface machining: tool path generation, tool orientation identification, and tool geometry selection. For each aspect, first concepts, requirements and fundamental research methods are briefly introduced. The major research methodologies developed in the past decade in each aspect are presented with details. Problems and future research directions are also discussed.	
	


\section{Overview of current issues in CNC} 

TO DO

\section{G-Codes Standards}

\subsection{NGC RS274D G-Codes}

\begin{tcolorbox}
TO DO - 

G-code (also RS-274 NGC), which has many variants, is the common name for the most widely used numerical control (NC) programming language. It is used mainly in computer-aided manufacturing to control automated machine tools. 

ISO 6983-1:2009 specifies requirements and makes recommendations for a data format for positioning, line motion and contouring control systems used in the numerical control of machines. ISO 6983-1:2009 helps the co-ordination of system design in order to minimize the variety of program manuscripts required, to promote uniformity of programming techniques, and to foster interchangeability of input programs between numerically controlled machines of the same classification by type, process, function, size and accuracy. It is intended that simple numerically controlled machines be programmed using a simple format, which is systematically extensible for more complex machines.

\end{tcolorbox}

\subsection{NURBS G-Codes}

Low end CAD/CAM systems do not generate NURBS G-Code.

Non-Uniform Rational B-Splines NURBS have been used by CAD systems for some time. That is why it seems so natural that CNCs should be able to employ tool paths that are also defined in terms of NURBS. However, most CNCs today instead require contoured tool paths to be defined using straight lines, or chords. 

And this long-practiced approach can lead to inefficiencies familiar to almost any die or mold shop. Using chords to define complex geometries accurately results in large, data-dense program files that historically have been difficult to manage and slow to execute. The development of NURBS-interpolating CNCs promised programs that could define the same complex geometries with fewer blocks of code, and thus could provide some relief for the data-flow bottlenecks.

But something happened along the way. CNCs became more powerful, and powerful CNCs became less costly. Compared to the best controls available only a short time ago, many new controls today offer superior networking capability, cheap memory and faster processing speed. These improvements mean enormous program files for complex workpieces are easier to deliver to the CNC, where they can be stored entirely instead of drip-fed. And once there, the programs can be executed more efficiently. Processing rates on the order of 1,000 blocks per second, combined with CNC look-ahead features to smooth out inertial effects, can let a new control today execute even a long series of very tiny chords fast enough to keep a machine moving accurately at a high feed rate.

Using nurbs interpolation requires not just a CNC capable of it, but also a CAM system able to output nurbs tool paths. And these CAM systems can use a variety of approximations to get from one set of nurbs to the other. Here are just two reasons why:

Surfaces are not tool paths. Mr. Arnone states it this way: "When a plane is intersected with a nurbs surface to create a tool path, a nurbs curve does not result." 

In other words, the software is still approximating to get to the tool path. And like any approximation, this one is governed by an accuracy band, comparable to chordal tolerance.

A straight line may be the route between two curves. Some CAM systems translate the CAD geometry into nurbs toolpaths by generating a series of chords first, then translating the chord tool paths into nurbs. For CAM systems that use this approach, there literally is a chordal tolerance at work.

Finally, it's worth noting that at the level where the cutting tool hits the workpiece, the movement is still in straight lines. The CAD model may be represented by nurbs, the CNC may read tool paths in terms of nurbs, but when the CNC communicates movement commands to the processor controlling a given axis, it does this by specifying a target point. That is, a target toward which the axis advances in a straight line.

====================
Optimized NURBS Based G-Code Part Program for High-Speed CNC Machining

August 2014 
DOI: 10.1115/DETC2014-34884

Conference: ASME 2014 International Design Engineering Technical Conferences and Computers and Information in Engineering Conference




\subsection{STEP-NC G-Codes}

\cite{STEP-NC_2018} STEP-Compliant CAD/CNC Systems for Feature-Oriented Machining\\

\begin{tcolorbox}

\cite{Yusri_2013} - In the projection toward the development of next generation CNC system, the major problem of current International Standards Organization (ISO) data interface model (ISO 6983) limitations and commercial CNC unit vendor specifications dependency were found in the CNC machines and systems. Later on, an ISO standard known as Standard for The Exchange of Product Data (STEP) or ISO 10303 was introduced to provide remedy for the problems of current data interface model limitations in Computer Aided Design (CAD)/Computer Aided Manufacturing (CAM) systems. After that successful implementation, the standard further extended to implement the STEP features on the CNC, for that a new standard known as STEP-Numeric Control (NC) or ISO 14649 was introduced. This standard has the abilities to achieve the aims of modern CNC systems. However, for enabling these facilities into the CNC machine, the machines need to be independent from any of the vendor specifications. Open Architecture Control (OAC) technology provides a more open environment to the CNC machine. In the race towards the development of next generation CNC, the combination of the STEP-NC and OAC technology became the hot topic of the research. However, in this work both ISO data interface model interpretation, its verification and execution has been highlighted with the introduction of the new virtual component technology based techniques. The system was composed of ISO data interpretation, 3D simulation and machine motion control modules. The system was also tested experimentally and found to be very satisfactory.

The development of STEP started in 1984 as a successor of IGES, but due to the complexity of the project, the initial standard was only published in 1994. Initial Graphics Exchange Specification (IGES ) is a neutral file format designed to transfer 2D and 3D drawing data between dissimilar CAD systems. The IGES standard defines two file formats: fixed-length ASCII, which stores information in 80-character records, and compressed ASCII.


\end{tcolorbox}

\section{Reference-Pulse interpolation}

In the Reference-Pulse interpolation method, reference signals from the computer are \\transmitted as a sequence of reference pulses, basically a sequence of external interrupts sent to the control loop of the CNC machine drive. In this interpolation method, the computer transmits a sequence of reference pulses to each axis-of-motion, where each pulse produces one BLU (Basic Length Unit) of movement. 
\vspace*{1\baselineskip}

Each axis-of motion is controlled by two pulsed lines, one for clockwise rotation (CW) and the other for counter-clockwise (CCW) rotation. The accumulated number of pulses for each axis represents the location of the tool on that axis. The pulse frequency, generated by varying the time spacing between pulses, is proportional to the velocity of the tool along the axis. This frequency represents the axis feedrate. 
\vspace*{1\baselineskip}

The reference-pulse method can be used for actuating stepper motors in an open-loop system or sent as reference signals to servo motors in a closed-loop control system. For example, a pulse train of varying frequency is output to the servo control module. For each pulse, the servo system for an axis causes an incremental movement along the axis. 
\vspace*{1\baselineskip}

All reference-pulse interpolations are iterative and is usually controlled by an adjustable interrupt clock. A single iteration of the routine is executed at each interrupt, and this produces the output pulse. For the computation, the maximum feedrate (feed pulses) generated is limited by the maximum attainable computer interrupt rate. The interrupt rate in turn depends on the computation time of the algorithm. 
\vspace*{1\baselineskip}

The faster the computation, the higher will be the feedrate. The number of iterations required to move the CNC machine to a certain distance on a contour path segment for each axis-of-motion is also calculated by the interpolation algorithm. 
\vspace*{1\baselineskip}

\begin{tcolorbox}
	
	
\cite{Smith_1993-01}	CNC Machining Technology
\cite{Smith_1993-02} CNC Controllers and Programming Techniques\\	
\cite{Koren_1981} Reference-Pulse circular interpolators for CNC Systems\\
\cite{Giap_2014} - A Reference-Pulse Generator for Motion Control System\\

\end{tcolorbox}

% ================================
\section{Reference-Word interpolation}

In the Reference-Word (Sampled-Data) interpolation method, the interpolator runs in an online iterative mode and generate words (sampled data) which are supplied as references to the running CNC control loops. It is a point-to-point control method used to move to the desired position between two successive interpolated points along each axis. The coordinate points to reach from the present position are computed for each iteration and the calculated data is transmitted to each axis in the form of sampled data. 
\vspace*{1\baselineskip}

For example in 2D motion, this calculated sample data consists of the next x and y coordinates and their velocities along each of the x and y axes. In a 2D contouring system the tool is cutting while the machine axes are moving. The contour path of the part is determined by the ratio between the velocities along the two axes. During a closed loop execution, the interpolation subroutines continuously provide destinations and velocity set points to the stepper or servo drive systems. 
\vspace*{1\baselineskip}

In contrast, in reference-word interpolation the maximum feedrate (velocity) is not limited by the execution speed of the processor. It is the stepper or servo system dynamics that limits the speed rather than the interpolator loop execution time.
\vspace*{1\baselineskip}

\begin{tcolorbox}
	
\cite{Koren_1978} Design Parameters for Sampled-Data Drives for CNC Machine Tools\\
\cite{Koren_1982} Reference-Word circular interpolators for CNC systems\\
\cite{Yang_2018}  Research and Development of Sampled-data Interpolation Algorithm Software in CNC System Based on the Visual C++ \\

\end{tcolorbox}

%% ==============================
\section{NURBS CNC Interpolations}

Some of the newer CNCs have NURBs interpolation available, but most controls won't have it unless you order it as an option. NURBs interpolation would let you write a G-code program with the NURBs data, which can make a single pass on a complex surface with just one block of G-code data. It reduces the size of the G-code file, and it also increases the maximum feedrate you can achieve.

FANUC USA uses G06.1 G-Code.

====================================
G05 P10000 	High-precision contour control (HPCC) 	M 	  	Uses a deep look-ahead buffer and simulation processing to provide better axis movement acceleration and deceleration during contour milling

G05.1 Q1. 	AI Advanced Preview Control 	M 	  	Uses a deep look-ahead buffer and simulation processing to provide better axis movement acceleration and deceleration during contour milling

G06.1 	Non-uniform rational B-spline (NURBS) Machining 	M 	  	Activates Non-Uniform Rational B Spline for complex curve and waveform machining (this code is confirmed in Mazatrol 640M ISO Programming) 

=====================================
ed (I just got the DXF), but you might be able to apply that reasoning to surfacing. And if you can create a CAD model, CAM it from there. 


\begin{tcolorbox}

Real-time NURBS curve interpolator for 5-Axis CNC machining

February 2018

DOI: 10.7736/KSPE.2018.35.2.129

Sungchul Jee


\cite{FarinBook_1999} - The first appearance of NURBS is in K. Vesprille's PhD thesis [151],
where he makes heavy use of homogeneous coordinates, an approach that
goes back to S. Coons and R. Forrest. Riesenfeld [130] realized early that
using homogenous coordinates mandates working with projective geometry. This book follows these ideas and attempts to base the theory of NURBS firmly in projective geometry. This way, we gain more insight into
theoretical issues as well as practical ones. After a general outline of projective geometry, we introduce conics
through a classical projective definition. Using the concept of cross ratios, we arrive at the de Casteljau algorithm for conics. Then we cover areas such as rational Bezier curves curves, NURBS curves and surfaces, triangular patches, Gregory patches, and more. The book closes with some practical examples, including a discussion of the IGES NURBS data specifications.

\cite{Yang_1994} - Parametric interpolator versus linear interpolator for precision surface machining\\
\cite{Zhang_2009} - Development of a NURBS Curve Interpolator with Look-ahead Control and Feedrate Filtering for CNC System\\
\cite{Bahr_2001} A real-time scheme of cubic parametric curve interpolations for CNC systems

\cite{Cho_2017} Feedrate fluctuation compensating NURBS interpolator for CNC machining\\

\cite{ErrorBook_2016} - Accuracy and Error Compensation of CNC Machining Systems\\
\cite{Liangji_2009} - Computer Numerical Controlled System with NURBS Interpolator\\
\cite{Lin_2007-02} - Development of Real-time Look-Ahead Algorithm for NURBS Interpolator with Consideration of Servo Dynamics\\
\cite{Lin_2007-01}	Development of a dynamics-based NURBS interpolator with real-time look-ahead algorithm\\

The International Journal of Advanced Manufacturing Technology

April 2008, Volume 36, Issue 9 --10, pp 927 -- 935  Cite as
The design of a NURBS pre-interpolator for five-axis machining

Authors
Authors and affiliations

Wei LiEmail authorYadong LiuKazuo YamazakiMakoto FujisimaMasahiko Mori

==========================================
https://www.wittystore.com/g-code-commands
==========================================
G06.1 	Non-uniform rational B-spline(NURBS) Machining 	M 	  	

Activates Non-Uniform Rational B Spline for complex curve and waveform machining (this code is confirmed in Mazatrol 640M ISO Programming)


===============================================
List of G-Codes commonly found on FANUC and similarly designed controls
Published In: Tech Explained Hits: 800

G-codes, also called preparatory codes, are any word in a CNC program that begins with the letter G. Generally it is a code telling the machine tool what type of action to perform, such as:

Rapid movement (transport the tool as quickly as possible in between cuts)
Controlled feed in a straight line or arc
Series of controlled feed movements that would result in a hole being bored, a workpiece cut (routed) to a specific dimension, or a profile (contour) shape added to the edge of a workpiece
Set tool information such as offset
Switch coordinate systems



==========================================
Warning: G5.2, G5.3 is experimental and not fully tested.

G5.2 is for opening the data block defining a NURBS and G5.3 for closing the data block. In the lines between these two codes the curve control points are defined with both their related weights (P) and the parameter (L) which determines the order of the curve.

The current coordinate, before the first G5.2 command, is always taken as the first NURBS control point. To set the weight for this first control point, first program G5.2 P- without giving any X Y.

The default weight if P is unspecified is 1. The default order if L is unspecified is 3.


\end{tcolorbox}

%=========================
\section{Contouring control and Error Compensation}

\begin{tcolorbox}
	
	
\cite{Zheng_2014} The Error Analysis of Surface Machining on the {CNC} Machine Tools\\
\cite{Chen_1997} Accuracy improvement of three-axis CNC machining centers by quasi-static error compensation\\
\cite{Dong_2013} Adaptive Error Control Algorithm for High Speed Two-Axis {CNC} Contouring\\
\cite{Ma_2017}	Reduction of the Contouring Error in High-Feed-Speed Machining by Real-Time Tracking-Error Compensation\\
\cite{Zuo_2013}	Integrated Geometric Error Compensation of Machining Processes on CNC Machine Tool\\
\cite{Hendrawan_2018} Embedded Iterative Learning Contouring Controller Based on Precise Estimation of Contour Error for CNC Machine Tools\\	
\cite{Guo_2013} Tracking error reduction in CNC machining by reshaping the kinematic trajectory\\
\cite{Chen_2017} Research on Error Compensation Technology for CNC Machining\\
\cite{Wang_2012} Contour error and control algorithm in CNC machining tool\\	
\cite{Schmitz_2000} Dynamic evaluation of spatial CNC contouring accuracy\\
\cite{ErrorBook_2016} Accuracy and Error Compensation of CNC Machining Systems\\
\cite{Guo_2013} Tracking error reduction in CNC machining by reshaping the kinematic trajectory\\
\cite{Zuo_2013} Integrated Geometric Error Compensation of Machining Processes on CNC Machine Tool
\cite{Weihua_Li_2010} Develop on feed-forward real time compensation control system for movement error in CNC machining\\
\cite{Lau_MSThesis_2005} CNC machining accuracy enhancement by tool path compensation method\\


\end{tcolorbox}


%=========================
\section{Look-ahead control}

Conventional CNC systems only provide line (G01) and circular (G02, G03) interpolations, so the CAD/CAM systems have to divide the curves into a large number of small linear and circular arc segments while maintaining the set of contour error limits. 
\vspace*{1\baselineskip}

Due to the short segments, most of the time spent in interpolation is on stop-start motions, typically accelerating, decelerating, or pausing between instructions. The frequent starts and stops for very small linear motions for every segment causes jerks during interpolation, especially at the junctions of the segments. This results in discontinuous feedrate (velocity) profiles, jerky and unsmooth overall machining process. 
\vspace*{1\baselineskip}

The feedrate look-ahead control algorithm is introduced to deal with sudden changes of feedrate [13-15,23-25]. Look-ahead control is a pre-processing task of the path contour before the real machining. The main goal is to achieve a smooth feedrate profile by looking at deceleration regions in the path. 
\vspace*{1\baselineskip}

The feedrate look-ahead control algorithm has three tasks: (1) to detect the deceleration point timely before reaching a linking corner or terminal point, (2) to smoothen and re-plan the feedrate in the deceleration region, and (3) to make the deceleration (braking) meet the capabilities of the physical machine like electrical motors. Braking is very important especially for high speed machining because exceeding the braking capability of the machine can cause unwanted machine vibrations. 
\vspace*{1\baselineskip}

The look-ahead interpolation algorithm is executed such that the processing error lies within a limited range. The look-ahead control strategy can be performed as either an off-line prediction, or as an on-line computing operation [24]. 
\vspace*{1\baselineskip}

Many methods have been proposed for look-ahead control strategies. Some examples are:

\begin{enumerate}
	\item Generate a recursive trajectory to estimate and determine the deceleration stage according to the distance left to travel on the segment. 
	
	\item Apply a hybrid digital convolution technique through a look-ahead scheme that smoothed the feedrate between the joint of two curve segments. 
	
	\item Implement an integrated look-ahead dynamics-based algorithm by considering contour and servo errors simultaneously.
	
	\item Introduce a look-ahead trajectory generation to determine the acceleration stage according to the fast estimated arc length and the reverse interpolation of each curve segment. 
	
	\item Create a real-time look-ahead scheme that combines of path-smoothing, bi-directional scanning and feedrate scheduling.
\end{enumerate}
% \vspace*{1\baselineskip}

\begin{tcolorbox}
	
\cite{Lin_2007-02} - Development of Real-time Look-Ahead Algorithm for NURBS Interpolator with Consideration of Servo Dynamics	

\cite{Lin_2007-01}	Development of a dynamics-based NURBS interpolator with real-time look-ahead algorithm

\cite{Zhang_2009} - Development of a NURBS Curve Interpolator with Look-ahead Control and Feedrate Filtering for CNC System
	
\cite{Tingting_2016} - This paper presents a perfect tracking method by combining iterative learning control (ILC) with disturbance observer (DOB) for CNC machine tools that perform the same tasks repeatedly. Although CNC systems have many nonlinear factors, they can be easily solved by ILC instead of complicated modeling. Also, for repeated disturbance, ILC performs very well and for non-repeated disturbance, DOB works much better. Besides, in order to facilitate the application, this paper proposes a variable gain algorithm with conditional compensation. Simulative results demonstrate the proposed learning scheme can improve machining accuracy when the CNC machine tools perform repetitive machining tasks.
\end{tcolorbox}

\section{Feedrate Filtering}

The function of feedrate filtering is to obtain a continuous acceleration profile and avoid sudden changes in acceleration/deceleration during machining. The function of the adaptive feedrate adjustment is to adjust the feedrate, confine chord error and make acceleration meet the capabilities of the machine.
\vspace*{1\baselineskip}

To obtain a continuous feedrate and acceleration during the whole interpolation process, one method is feedrate filtering [13]. This method adaptively adjusts the feedrate according to the different curvatures to meet the demand of the CNC machining accuracy. 
\vspace*{1\baselineskip}

Another popular method to eliminate jerky motion is spline interpolation [13-18]. In spline interpolation, the tool moves smoothly from one point to another. For example, in Non-Uniform Rational B-Spline (NURBS) interpolation, the curve is parametrized and approximated by a set of interpolated data points, consisting of control points, knot points and weights. 
\vspace*{1\baselineskip}

NURBS parametric interpolation is often preferred over conventional linear and circular
interpolator for its merits in terms of the model representation, feedrate smoothness and application range. The parametric curve is smooth and continuous so feedrate continuity is achieved effectively since the sharp junctions between curve segments are avoided.
\vspace*{1\baselineskip}

Other methods include constant feedrate interpolation algorithms based on first-order, second-order Taylor expansions, and adaptive interpolation algorithm with confined chord error [10,12,22]. A feedrate fluctuation compensating algorithm that predicts and compensates feedrate fluctuation in real-time, has also been considered [18].
\vspace*{1\baselineskip}

\begin{tcolorbox}


\cite{Weihua_Li_2010} Develop on feed-forward real time compensation control system for movement error in CNC machining\\
	
\cite{Cho_2017} Feedrate fluctuation compensating NURBS interpolator for CNC machining\\
\cite{Zhang_2009} - Development of a NURBS Curve Interpolator with Look-ahead Control and Feedrate Filtering for CNC System\\

\end{tcolorbox}

% ==============================
\section{Realtime and Parallel Computing}

\begin{tcolorbox}
	
\textbf{Parallel}\\
\cite{Hongya_2016} A parallel CNC system architecture based on Symmetric Multi-processor\\ 




\textbf{Realtime}\\

\cite{Weihua_Li_2010} Develop on feed-forward real time compensation control system for movement error in CNC machining\\
\cite{Bahr_2001} A real-time scheme of cubic parametric curve interpolations for CNC systems\\
\cite{Lin_2007-02} - Development of Real-time Look-Ahead Algorithm for NURBS Interpolator with Consideration of Servo Dynamics\\	
\cite{FYP_Abzal_2012} C/C++ and Python for Linux Realtime Parallel Port Software Driver \\
\cite{Lin_2007-01}	Development of a dynamics-based NURBS interpolator with real-time look-ahead algorithm\\
\cite{Shpiltani_1994} Real-time curve interpolators\\
\cite{Bahr_2001} A real-time scheme of cubic parametric curve interpolations for CNC systems\\
\cite{Ma_2017}	Reduction of the Contouring Error in High-Feed-Speed Machining by Real-Time Tracking-Error Compensation\\

\end{tcolorbox}

%% =================
Various researchers have implemented realtime CNC software systems on Windows platform running its RTX realtime operating system [26]. Free and open source CNC software are also available for the Linux platform through RT-Linux [27] realtime operating system (RTOS) or Real-Time Application Interface (RTAI) extensions. 
\vspace*{1\baselineskip}

The LinuxCNC [31] system is an open source RTAI-based CNC machine controller. It can drive milling machines, lathes, 3d printers, laser cutters, plasma cutters, robot arms, hexapods, and more. The controller for LinuxCNC is named EMC2 (Enhanced Machine Controller version 2).
\vspace*{1\baselineskip}

To take advantage of multi-cores and multi-processors available on personal computers, several implementations for high-speed CNC interpolation scheme using parallel computing have also been conducted. 
\vspace*{1\baselineskip}

In one method, the CNC interpolation method was divided into two tasks, the rough task executing in the personal computer and the fine task in the I/O card. During the interpolation procedure, double data buffers were constructed to exchange interpolation data between the two tasks in parallel [28].
\vspace*{1\baselineskip}

In another method, parallelization of the CNC algorithm was implemented using multi-\\ threading based on the Pthread software programming library [29]. Another variation is the method where a parametric curve realtime CNC interpolator processing was executed in parallel using the C++ Open Multi-Processing (OpenMP) Application Program Interface (API) library [30].
\vspace*{1\baselineskip}

% ====================================
\section{AI Approaches in CNC Machining}

\cite{Hendrawan_2018} Embedded Iterative Learning Contouring Controller Based on Precise Estimation of Contour Error for CNC Machine Tools\\
\cite{Tingting_2016} The design of iterative learning control scheme for CNC machine tools\\
\cite{Koren_1993} CNC Interpolators: Algorithms and Analysis\\
\cite{Koren_1976} Interpolator for a CNC System\\
\cite{Zheng_2014} The Error Analysis of Surface Machining on the {CNC} Machine Tools\\
\cite{Dong_2013} Adaptive Error Control Algorithm for High Speed Two-Axis {CNC} Contouring\\

% =====================================
\section{Commercial versus Open CNC systems}

\cite{Yusri_2013} Frame Work of LV-UTHM: AN ISO 14649 Based Open Control System for CNC Milling Machine\\


% =================================
\section{Embedded devices in CNC systems}

\cite{FYP_Asyrul_2017} Using Raspberry Pi 3 Model B to drive a CNC System in Real Time\\
\cite{FYP_Abzal_2012} C/C++ and Python for Linux Realtime Parallel Port Software Driver\\
\cite{FYP_Hazmi_2014} Using Arduino Due to drive a CNC System\\
\cite{FYP_Charles_2014} Using the Nexys-3 Spartan-6 FPGA board to develop a closed-loop feedback CNC system\\
\cite{Education_1993} A CNC machining system for education\\
\cite{Kats_2017} On the approach to CNC machining simulation improving\\

\cite{Hendrawan_2018} Embedded Iterative Learning Contouring Controller Based on Precise Estimation of Contour Error for CNC Machine Tools\\

% ==========================================
\section{Summary on Literature Survey}
% ==========================================

