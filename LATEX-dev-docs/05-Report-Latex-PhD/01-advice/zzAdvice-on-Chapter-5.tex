\pagebreak
%% ===========================================
\begin{tcolorbox}
	
\section{zzAdvice-on-Chapter-5}

\textbf{RESEARCH IMPLEMENTATION}
\vspace*{1\baselineskip}

\textbf{Chapter 5 Guidelines - TO REMOVE LATER}	
\vspace*{1\baselineskip}	
	
\subsection{Implementation Schedule - Gantt Chart}

Depth of plans - Not all research proposals lend themselves easily to the creation of detailed work plans. In some cases, namely when the work fits the broader plans of a research group that is progressing steadily, it is possible do build a detailed description of 

what the researcher plans to do (literature to explore in depth, principles or theorems to formulate and prove, experiments to carry out, sub-systems to build, systems integrations to perform, tests to accomplish).

Define and List the tasks

\subsection{Critical Tasks} 

Identify Critical Tasks, why critical. Focus. Plan to tackle problems. The plan should anticipate the problems likely to be found along the way and describe the approaches to be followed in solving them.

\subsection{Research Milestones}

Plan the tasks - In these cases, it is possible, and desirable, to establish specific milestones and time lines and a Gantt diagram. Identify critical tasks.

\subsection{Publications Plan}

Plan for publications - It should also anticipate the conferences and journals to which the work in progress is expected to be submitted along the way, and schedule it in a Goals for Publication section of the work plan.

Focus - In spite of its contingency, this list may work marvels in keeping the researcher focused, motivated and beneficially under pressure.

\section{Expected Results}

\textbf{Result oriented} - Whatever its nature, comprehensive or sketchy, your work plan should be able to put in perspective the implications of the successive steps of your work, reinforcing, in the mind of the reader, the conviction that your approach is solidly oriented toward results, that the topic is timely and relevant, and that the outcomes of the project will contribute significantly to the enhancement of the field.

	
	
\end{tcolorbox}

