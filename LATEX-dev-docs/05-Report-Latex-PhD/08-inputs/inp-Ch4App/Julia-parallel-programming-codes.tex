%%% LANDSCAPE BEGINS =======================================
\pagebreak
\clearpage
%\begin{flushleft}
%\begin{landscape}

% =========================================================
\subsection{App4-Julia Parallel Programming Codes}
% =========================================================

\begin{enumerate}

	\item \textbf{Parallel computing in Julia}
	
	\item \url{https://github.com/JuliaLang/julia/blob/master/doc/src/manual/parallel-computing.md}
	
	\item For newcomers to multi-threading and parallel computing it can be useful to first appreciate the different levels of parallelism offered by Julia. We can divide them in three main categories :
	
	\begin{enumerate}
		\item Julia Coroutines (Green Threading)
		\item Multi-Threading
		\item Multi-Core or Distributed Processing
	\end{enumerate}
	
	\item We will first consider Julia [Tasks (aka Coroutines)](@ref man-tasks) and other modules that rely on the Julia runtime library, that allow to suspend and resume computations with full control of inter-Tasks communication without having to manually interface with the operative system's scheduler. 
	
	\item Julia also allows to communicate between Tasks through operations like wait and fetch. Communication and data synchronization is managed through Channels, which are the conduit that allows inter-Tasks communication.
	
	\item Julia also supports experimental multi-threading, where execution is forked and an anonymous function is run across all threads. Described as a fork-join approach, parallel threads are branched off and they all have to join the Julia main thread to make serial execution continue. 
	
	\item Multi-threading is supported using the Base.Threads module that is still considered experimental, as Julia is not fully thread-safe yet. In particular segfaults seem to emerge for I\O operations and task switching. As an un up-to-date reference, keep an eye on the issue tracker. Multi-Threading should only be used if you take into consideration global variables, locks and atomics, so we will explain it later.
	
	\item In the end we will present Julia's way to distributed and parallel computing. With scientific computing in mind, Julia natively implements interfaces to distribute a process through multiple cores or machines. Also we will mention useful external packages for distributed programming like MPI.jl and DistributedArrays.jl.
	
	\item \textbf{Coroutines and Channels in Julia}

	\item Julia's parallel programming platform uses [Tasks (aka Coroutines)](@ref man-tasks) to switch among multiple computations. 
	
	\item To express an order of execution between lightweight threads communication primitives are necessary. 
	
	\item Julia offers Channel(func::Function, ctype=Any, csize=0, taskref=nothing) that creates a new task from func, binds it to a new channel of type ctype and size csize and schedule the task. 
	
	\item Channels can serve as a way to communicate between tasks, as Channel{T}(sz::Int) creates a buffered channel of type T and size sz. 
	
	\item Whenever code performs a communication operation like fetch or wait, the current task is suspended and a scheduler picks another task to run. A task is restarted when the event it is waiting for completes.
	
	\item For many problems, it is not necessary to think about tasks directly. However, they can be used to wait for multiple events at the same time, which provides for dynamic scheduling. In dynamic scheduling, a program decides what to compute or where to compute it based on when other jobs finish. This is needed for unpredictable or unbalanced workloads, where we want to assign more work to processes only when they finish their current tasks.
	
	\item The section on Tasks in Control Flow discussed the execution of multiple functions in a co-operative manner. Channels can be quite useful to pass data between running tasks, particularly those involving I/O operations.
	
	\item Examples of operations involving I/O include reading/writing to files, accessing web services, executing external programs, etc. In all these cases, overall execution time can be improved if other tasks can be run while a file is being read, or while waiting for an external service/program to complete.
	
	\item A channel can be visualized as a pipe, i.e., it has a write end and a read end :
	\begin{enumerate}
		\item Multiple writers in different tasks can write to the same channel concurrently via put! calls.
		
		\item Multiple readers in different tasks can read data concurrently via take! calls.
	\end{enumerate}
	
	
	\item Channels are created via the Channel{T}(sz) constructor. The channel will only hold objects of type T. If the type is not specified, the channel can hold objects of any type. sz refers to the maximum number of elements that can be held in the channel at any time. For example, Channel(32) creates a channel that can hold a maximum of 32 objects of any type. A Channel{MyType}(64) can hold up to 64 objects of MyType at any time.
	
	\item If a Channel is empty, readers (on a take! call) will block until data is available.
	
	\item If a Channel is full, writers (on a put! call) will block until space becomes available.
	
	\item isready tests for the presence of any object in the channel, while wait waits for an object to become available.
	
	\item A Channel is in an open state initially. This means that it can be read from and written to freely via take! and put! calls. close closes a Channel. On a closed Channel, put! will fail.
	
	\item The current version of Julia multiplexes all tasks onto a single OS thread. Thus, while tasks involving I/O operations benefit from parallel execution, compute bound tasks are effectively executed sequentially on a single OS thread. 
	
	\item Future versions of Julia may support scheduling of tasks on multiple threads, in which case compute bound tasks will see benefits of parallel execution too.
	

	
\end{enumerate}

% ==========================================================
\lstset{basicstyle=\footnotesize, numberstyle=\tiny\color{blue}, frame=single, numbers=left, firstnumber=1, stepnumber=1, numbersep=1pt, xleftmargin=2.0em, framexleftmargin=1.5em, xrightmargin=0.0em, breaklines=true, breakatwhitespace=false, breakindent=5pt, prebreak=\space, postbreak=\space }
% ==========================================================

\begin{lstlisting}[caption={App4-Julia Parallel Programming Codes}, label=App4-Julia Parallel Programming Codes]
# File: julia-script-04.jl
# Date: Fri Aug 31 16:02:56 +08 2018
# Author: WRY
# ==========================================================
# Comparing sequential and parallel code execution.
# PARALLEL COMPUTING IN JULIA LANGUAGE, developed at MIT, USA.
# ==========================================================

	using Dates 
	print(Dates.time(), " Current date today() = ", Dates.today(), "\n")
	print(Dates.time(), " Current date time now: ", Dates.now(), "\n")
	print(Dates.time(), " Bismillah from WRY in Julia script. \n") 
	
	using PyCall
	@pyimport psutil
	nCPU = psutil.cpu_count()
	print(Dates.time(), " PyCall: Value of psutil.cpu_count() = $nCPU \n")

	using Hwloc
	topology = Hwloc.topology_load()
	counts = Hwloc.histmap(topology)
	ncores = counts[:Core]
	npus = counts[:PU]
	
	print(Dates.time(), " Hwloc: This machine has $ncores cores and $npus PUs (processing units). \n")
	nphysicalcores = Hwloc.num_physical_cores()
	print(Dates.time(), " Hwloc: This machine has $nphysicalcores physical cores. \n")

# ==========================================================
# PARALLEL COMPUTING IN JULIA
# ==========================================================
# Create n workers at start of Julia session
# MUST RUN "julia -p n" TO GET nprocs(), IF NOT ERROR

# View number of workers + master process
	my_nprocs = nprocs();
	print(Dates.time(), " Number of workers + master process = $my_nprocs \n")

# Create n workers during a session if required
	# addprocs(n);

# workers are addressed by numbers (PIDs)
# master process had PID =1, the rest are PID of workers
	my_wpid = workers();
	print(Dates.time(), " List of workers PIDs = $my_wpid \n")

# ==========================================================
function fxn02(input02, sleeprand02)
# ==========================================================
	print(Dates.time(), " Started  running fxn02 \n")
	print(Dates.time(), " Value input02 = $input02 \n")
	print(Dates.time(), " Value sleeprand02 = $sleeprand02 \n")
	sleep(sleeprand02)
	str02 = " Result $input02 returned from fxn02"  
	print(Dates.time(), " Finished running fxn02 \n")
	return (str02)	
end 

# ==========================================================
function fxn03(input03, sleeprand03)
# ==========================================================
	print(Dates.time(), " Started  running fxn03 \n")
	print(Dates.time(), " Value input03 = $input03 \n")
	print(Dates.time(), " Value sleeprand03 = $sleeprand03 \n")
	sleep(sleeprand03)
	str03 = " Result $input03 returned from fxn03"  
	print(Dates.time(), " Finished running fxn03 \n")
	return (str03)	
end 

% ==========================================================
function fxn04(input04, sleeprand04)
# ==========================================================
	print(Dates.time(), " Started  running fxn04 \n")
	print(Dates.time(), " Value input04 = $input04 \n")
	print(Dates.time(), " Value sleeprand04 = $sleeprand04 \n")
	sleep(sleeprand04)
	str04 = " Result $input04 returned from fxn04"  
	print(Dates.time(), " Finished running fxn04 \n")
	return (str04)	
end 

# ==========================================================
function fxn05(input05, sleeprand05)
# ==========================================================
	print(Dates.time(), " Started  running fxn05 \n")
	print(Dates.time(), " Value input05 = $input05 \n")
	print(Dates.time(), " Value sleeprand05 = $sleeprand05 \n")
	sleep(sleeprand05)
	str05 = " Result $input05 returned from fxn05"  
	print(Dates.time(), " Finished running fxn05 \n")
	return (str05)	
end 

# ==========================================================
# DATA ASSIGNED TO FUNCTIONS
	input02 = 2.2222; sleeprand02 = rand(1:10)
	input03 = 3.3333; sleeprand03 = rand(1:10)
	input04 = 4.4444; sleeprand04 = rand(1:10)
	input05 = 5.5555; sleeprand05 = rand(1:10)
	
# ==========================================================
# SEQUENTIAL FUNCTIONS EXECUTION
# ==========================================================

	print(Dates.time(), " ====== START SEQUENTIAL FUNCTIONS EXECUTION ====== \n")
	runfxn02 = fxn02(input02, sleeprand02)
	runfxn03 = fxn03(input03, sleeprand03)
	runfxn04 = fxn04(input04, sleeprand04)
	runfxn05 = fxn05(input05, sleeprand05)

# DISPLAY RESULTS OF EXECUTIONS
	print(Dates.time(), " In Main runfxn02 = $runfxn02 \n")
	print(Dates.time(), " In Main runfxn03 = $runfxn03 \n")
	print(Dates.time(), " In Main runfxn04 = $runfxn04 \n")
	print(Dates.time(), " In Main runfxn05 = $runfxn05 \n")

# =========================================================
# PARALLEL FUNCTIONS EXECUTIONS (ASYNCHRONOUSLY, OR NON BLOCKING)
# =========================================================

	print(Dates.time(), " ====== START PARALLEL FUNCTIONS EXECUTION ====== \n")
	const jobs = Channel{Int}(32);
	const results = Channel{Tuple}(32);

# ==========================================================
# DO WORK
# ==========================================================
	function do_work()
		for job_id in jobs
			print(Dates.time(), " job_id $job_id started do_work().\n")
			exec_time = rand(1:10)
			
			# DO SOMETHING USEFUL HERE
			sleep(exec_time)    # SIMULATION FOR ELAPSED TIME DOING REAL WORK
			
			print(Dates.time(), " job_id $job_id set to finish in $exec_time seconds. \n")
			put!(results, (job_id, exec_time))
			print(Dates.time(), " job_id $job_id finished do_work(). \n")
		end
	end;

============================================================
# MAKE JOBS
# ==========================================================
	function make_jobs(n)
		for i in 1:n
			put!(jobs, i)
		end
	end;

# ==========================================================
# FEED THE JOBS CHANNELS WITH "n" JOBS
# ==========================================================
	n = 4;
	@async make_jobs(n); 

# ==========================================================
# START 4 TASKS TO PROCESS REQUESTS IN PARALLEL
# ==========================================================
	for i in 1:4 
		@async do_work()
	end

# ==========================================================
# print out results
# ==========================================================
	@elapsed while n > 0 
		job_id, exec_time = take!(results)
		print(Dates.time(), " job_id $job_id actually finished in $(round(exec_time; digits=4)) seconds \n")
		global n = n - 1
	end

# =========================================================
print(Dates.time(), " Alhamdulillah from WRY in Julia script. \n")
# =========================================================

ALHAMDULILLAH 3 TIMES. WRY.
"""
EXECUTION 1

wruslan@HP-ELBook8470p-ub1604-64b:~/Downloads/temp julia -p 4 julia-script04.jl
1.535760901044821e9 Current date today() = 2018-09-01
1.535760901176391e9 Current date time now: 2018-09-01T08:15:01.176
1.535760901266606e9 Bismillah from WRY in Julia script. 
1.535760910499787e9 PyCall: Value of psutil.cpu_count() = 4 
1.535760918305807e9 Hwloc: This machine has 2 cores and 4 PUs (processing units). 
1.535760918463921e9 Hwloc: This machine has 2 physical cores. 
1.535760918464832e9 Number of workers + master process = 5 
1.535760918473897e9 List of workers PIDs = [2, 3, 4, 5] 
1.535760918762892e9 ====== START SEQUENTIAL FUNCTIONS EXECUTION ====== 
1.5357609187739e9 Started  running fxn02 
1.535760918774005e9 Value input02 = 2.2222 
1.535760918774057e9 Value sleeprand02 = 1 
1.535760919776272e9 Finished running fxn02 
1.535760919798129e9 Started  running fxn03 
1.535760919798278e9 Value input03 = 3.3333 
1.535760919798374e9 Value sleeprand03 = 2 
1.535760921801578e9 Finished running fxn03 
1.535760921820814e9 Started  running fxn04 
1.535760921820915e9 Value input04 = 4.4444 
1.535760921820977e9 Value sleeprand04 = 1 
1.535760922823194e9 Finished running fxn04 
1.535760922847947e9 Started  running fxn05 
1.535760922848116e9 Value input05 = 5.5555 
1.535760922848213e9 Value sleeprand05 = 3 
1.535760925852463e9 Finished running fxn05 
1.53576092585304e9 In Main runfxn02 =  Result 2.2222 returned from fxn02 
1.535760925853588e9 In Main runfxn03 =  Result 3.3333 returned from fxn03 
1.535760925854025e9 In Main runfxn04 =  Result 4.4444 returned from fxn04 
1.535760925854435e9 In Main runfxn05 =  Result 5.5555 returned from fxn05 
1.535760925854789e9 ====== START PARALLEL FUNCTIONS EXECUTION ====== 
1.535760926140693e9 job_id 1 started 
1.535760926140750e9 job_id 2 started 
1.535760926140759e9 job_id 3 started 
1.535760926140762e9 job_id 4 started 
1.535760930146089e9 job_id 2 set to finish in 4 seconds. 
1.535760930146177e9 job_id 4 set to finish in 4 seconds. 
1.535760930146448e9 job_id 2 actually finished in 4.0 seconds 
1.535760930178436e9 job_id 4 actually finished in 4.0 seconds 
1.535760932143593e9 job_id 3 set to finish in 6 seconds. 
1.535760932143825e9 job_id 3 actually finished in 6.0 seconds 
1.535760935145178e9 job_id 1 set to finish in 9 seconds. 
1.535760935145356e9 job_id 1 actually finished in 9.0 seconds 
1.535760935145937e9 Alhamdulillah from WRY in Julia script. 
wruslan@HP-ELBook8470p-ub1604-64b:~/Downloads/temp

ANALYSIS 1.1
1.535760,935,145,356e9  job_id 1 actually finished in 9.0 seconds 
-1.535760,926,140,693e9 job_id 1 started 
========================
9,004,663e9 CONFIRMED 9 seconds, 

ANALYSIS 1.2

1.535760932,143,825e9  job_id 3 actually finished in 6.0 seconds 
-1.535760926,140,759e9 job_id 3 started 
========================
6,003,066e9 CONFIRMED 6 seconds.

EXECUTION 2

wruslan@HP-ELBook8470p-ub1604-64b:~/Downloads/temp julia -p 4 julia-script04.jl
1.535761821515869e9 Current date today() = 2018-09-01
1.535761821656556e9 Current date time now: 2018-09-01T08:30:21.656
1.535761821745387e9 Bismillah from WRY in Julia script. 
1.535761830903611e9 PyCall: Value of psutil.cpu_count() = 4 
1.535761838714373e9 Hwloc: This machine has 2 cores and 4 PUs (processing units). 
1.535761838876378e9 Hwloc: This machine has 2 physical cores. 
1.535761838876955e9 Number of workers + master process = 5 
1.535761838901269e9 List of workers PIDs = [2, 3, 4, 5] 
1.535761839174190e9 ====== START SEQUENTIAL FUNCTIONS EXECUTION ====== 
1.535761839185062e9 Started  running fxn02 
1.535761839185163e9 Value input02 = 2.2222 
1.535761839185216e9 Value sleeprand02 = 7 
1.535761846192202e9 Finished running fxn02 
1.535761846214519e9 Started  running fxn03 
1.535761846214672e9 Value input03 = 3.3333 
1.535761846214757e9 Value sleeprand03 = 4 
1.535761850219946e9 Finished running fxn03 
1.535761850239776e9 Started  running fxn04 
1.535761850239921e9 Value input04 = 4.4444 
1.535761850240005e9 Value sleeprand04 = 10 
1.535761860247521e9 Finished running fxn04 
1.535761860269229e9 Started  running fxn05 
1.535761860269385e9 Value input05 = 5.5555 
1.535761860269468e9 Value sleeprand05 = 9 
1.535761869279662e9 Finished running fxn05 
1.535761869280255e9 In Main runfxn02 =  Result 2.2222 returned from fxn02 
1.535761869280763e9 In Main runfxn03 =  Result 3.3333 returned from fxn03 
1.535761869281252e9 In Main runfxn04 =  Result 4.4444 returned from fxn04 
1.535761869281725e9 In Main runfxn05 =  Result 5.5555 returned from fxn05 
1.535761869282145e9 ====== START PARALLEL FUNCTIONS EXECUTION ====== 
1.535761869574609e9 job_id 1 started do_work().
1.535761869574664e9 job_id 2 started do_work().
1.535761869574676e9 job_id 3 started do_work().
1.535761869574683e9 job_id 4 started do_work().
1.535761872579027e9 job_id 4 set to finish in 3 seconds. 
1.535761872579328e9 job_id 4 finished do_work(). 
1.535761872579457e9 job_id 4 actually finished in 3.0 seconds 
1.535761873576851e9 job_id 2 set to finish in 4 seconds. 
1.535761873577079e9 job_id 2 finished do_work(). 
1.535761873577144e9 job_id 2 actually finished in 4.0 seconds 
1.535761875577793e9 job_id 1 set to finish in 6 seconds. 
1.535761875578067e9 job_id 1 finished do_work(). 
1.535761875578148e9 job_id 1 actually finished in 6.0 seconds 
1.535761876576700e9 job_id 3 set to finish in 7 seconds. 
1.535761876576916e9 job_id 3 finished do_work(). 
1.535761876576973e9 job_id 3 actually finished in 7.0 seconds 
1.535761876577773e9 Alhamdulillah from WRY in Julia script. 
wruslan@HP-ELBook8470p-ub1604-64b:~/Downloads/temp 

ANALYSIS 2.1

1.535761,876,576,973e9 	job_id 3 actually finished in 7.0 seconds 
1.535761,876,576,916e9 	job_id 3 finished do_work(). 
-1.535761,869,574,676e9 job_id 3 started do_work().  
=======================
7,002,240e9 CONFIRMED 7 seconds.

"""
ALHAMDULILLAH 3 TIMES.
# ==========================================================
\end{lstlisting}

% =========================================================
%\end{landscape}
%\end{flushleft}
