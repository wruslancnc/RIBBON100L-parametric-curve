\pagebreak
%% ===========================================
\begin{tcolorbox}

\section{zzAdvice-on-Chapter-1}

\textbf{INTRODUCTION}
\vspace*{1\baselineskip}

\textbf{Chapter 1 Guidelines - TO REMOVE LATER}	
\vspace*{1\baselineskip}
	
\textbf{TO REMOVE LATER}: The  introduction \gls{apig} gives an overview of the research project you propose to carry out. It explains the background of the project, focusing briefly on the major issues of its knowledge domain and \gls{isa} clarifying why these issues are \gls{ide} worthy of attention. It then proceeds with the concise presentation of the research statement, which can take the form of a hypothesis, a research question, a project statement, or a goal statement.
\vspace*{1\baselineskip}

\textit{\textbf{TO REMOVE LATER}: The research statement should capture both the essence of the project and its delimiting boundaries, and should be followed by a clarification of the extent to which you expect its outcomes to represent an advance in the knowledge domain you have described.}
\vspace*{1\baselineskip}

\textit{\textbf{TO REMOVE LATER}: The introduction should endeavor, from the very beginning, to catch the reader's interest and should be written in a style that can be understood easily by any reader with a general science background. It should cite all relevant references pertaining to the major issues described,  and it should close with a brief description of each one of the chapters that follow.}
\vspace*{1\baselineskip}

\textit{\textbf{TO REMOVE LATER}: Many authors prefer to postpone writing the Introduction till the rest of the document is finished. This makes a lot of sense, since the act of writing tends to introduces many changes in the plans initially sketched by the writer, so that it is only by the time the whole document is finished that the writer gets a clear view of how to construct an introduction that is, indeed, compelling.}
\vspace*{1\baselineskip}

% \vspace*{1\baselineskip}

\end{tcolorbox}

\begin{tcolorbox}
	
% glossary-extra ha ha ha \gls{cnc} for support vector machine.

\end{tcolorbox}

