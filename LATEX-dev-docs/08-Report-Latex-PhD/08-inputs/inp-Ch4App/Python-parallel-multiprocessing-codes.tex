%%% LANDSCAPE BEGINS =======================================
%\pagebreak
%\clearpage
%\begin{flushleft}
%\begin{landscape}

% =========================================================
\subsection{App4-Python Parallel Multiprocessing Codes}\label{sec:App4-Python-Multiprocessing}
% =========================================================

% ==========================================================
\lstset{basicstyle=\footnotesize, numberstyle=\tiny\color{blue}, frame=single, numbers=left, firstnumber=1, stepnumber=1, numbersep=1pt, xleftmargin=2.0em, framexleftmargin=1.5em, xrightmargin=0.0em, breaklines=true, breakatwhitespace=false, breakindent=5pt, prebreak=\space, postbreak=\space }
% ==========================================================

\begin{lstlisting}[caption={App4-Python Parallel Multiprocessing Codes}, label=App4-Python Parallel Multiprocessing Codes]

#! /usr/bin/env python

# File: 02_simple_python_multi_processing.py
# Date: Thu Nov 29 16:38:40 MYT 2012
# Author: WRY
# Envn: Linux wruslan-ubuntu1004-rtai 2.6.32-122-rtai #rtai SMP Tue Jul 27 12:44:07 CDT 2010 i686 GNU/Linux

"""
NOTES ABOUT THIS PYTHON CODE: 

(1) This program is about multi-processing implemented using the python programming language. It is an example showing how to use queues to feed tasks to a collection of worker processes running in parallel and then collect the results.

(2) Multi-processing programs "share nothing" in its environment. This means we must have mechanisms to exchange data, read and write data, etc, among the many simultaneously running processes. In this multi-processing program, we will use a "queue".

(3) We are now talking about "creating processes" in multi-processing, instead of "creating threads" in multi-threading.

(4) In this program, a "worker" means a child process. The "MainProcess" means the parent process. Both child and parent processes have a process ID. For example, a "pid" (for a child) and a "ppid" (for the parent). On the Linux terminal, run the command "ps -ef", and look for the ppid and the pid.

(5) To execute this program, the command is: "python 01_simple_python_multi_processing.py | sort ". It is important to put in the sort command. Try without it. This will give a time ordered display of results.

(6) What happens if you purposely omit the "sort" command? Do it and see for yourselves. Is there any order? Do this for all three programs given here.

(7) Can you explain why "FINISHED:  bla bla bla" and "ALHAMDULILLAH WRY." are not the last two(2) statements to be displayed in the results, even though they are the last two(2) closing statements sent to the computer?

(8) In this program, we have purposely "commented many print lines" to avoid confusion. In the next program we open them all. Go and run it, i.e.  02_simple_python_multi_processing.py.

(9) In the third program, we cut away all the comments, i.e. 03_simple_python_multi_processing.py, and you should see how short the actual program really is (may not be clear). Ha ha ha. Now, you will recall about SE documentation and SE coding standards. 

For more information go to : 
http://docs.python.org/2/library/multiprocessing.html#multiprocessing-examples

"""
# =========================================================
# IMPORT THE REQUIRED PYTHON MODULES
# =========================================================
import multiprocessing
import datetime
import time
import random
import os
from   multiprocessing import Process, Queue, current_process, freeze_support

# =========================================================
def worker(input, output):

	for func, args in iter(input.get, 'STOP'):
		worker_result = calculate(func, args)
		output.put(worker_result)

def calculate(func, args):
	result = func(*args)

	# Note that this calculate_result is NOT a print statement (think)
	calculate_result = "%18.12f %11s pid=%s," %(time.time() - globalStartTime, current_process().name,\	
						current_process().pid), \
						"Child %s%s=%s" %((func.__name__), args, result)

	return calculate_result

def mult(a, b):
	time.sleep(0.5*random.random())
	return a * b

def plus(a, b):
	time.sleep(0.5*random.random())
	return a + b

def run_test():

	TASKS1 = [(mult, (i, 7)) for i in range(10)]
	TASKS2 = [(plus, (i, 8)) for i in range(5)]
	
	task_queue = multiprocessing.Queue()
	done_queue = multiprocessing.Queue()

	for task in TASKS1:
		task_queue.put(task)
	
	procs = []
	for i in range(numProcesses):
	
		# Create processes by calling the python library
		p = multiprocessing.Process(target=worker, args=(task_queue, done_queue))
		p.start()
		procs.append(p)
	
	for i in range(len(TASKS1)):
		print "%18.12f %11s pid=%s," %(time.time() - globalStartTime, current_process().name,\
		current_process().pid), \
		"done_queue = ", done_queue.get()
	
	for task in TASKS2:
		task_queue.put(task)
	
	for i in range(len(TASKS2)):
		print "%18.12f %11s pid=%s," %(time.time() - globalStartTime, current_process().name,\
		current_process().pid), \
		"done_queue = ", done_queue.get()
	
	# Tell child processes to stop
	for i in range(numProcesses):
		task_queue.put('STOP')

# =========================================================
# MAIN PROGRAM ENTRY
# =========================================================
if __name__ == '__main__':

	globalStartTime = time.time() 
	print "%18.12f %11s pid=%s," %(time.time() - globalStartTime, current_process().name, current_process().pid), \
	"BISMILLAH WRY"
	print "%18.12f %11s pid=%s," %(time.time() - globalStartTime, current_process().name, current_process().pid), \
	"STARTING:", datetime.datetime.now(), "at program run time %s " %(time.time() - globalStartTime), "seconds."
	
	print "%18.12f %11s pid=%s," %(time.time() - globalStartTime, current_process().name, current_process().pid), \
	"Actual computer cpu_count()  = %d " % multiprocessing.cpu_count()
	
	# TRY 4, 6, 20, 25 or 3
	numProcesses = 15
	print "%18.12f %11s pid=%s," %(time.time() - globalStartTime, current_process().name, current_process().pid), \
	"Number of processes to create:", numProcesses
	
	print "%18.12f %11s pid=%s," %(time.time() - globalStartTime, current_process().name, current_process().pid), \
	"Begin multi-processing to run concurrent processes not concurrent threads."

	# The freeze_support() line can be omitted if the program is not going 
	# to be frozen to produce a Windows executable.
	
	# COMMENT THE NEXT LINE. USED ONLY FOR WINDOWS.
	# freeze_support()
	
	# THE RUN_TEST() ACTIVATES MULTI-PROCESSING
	run_test()
	
	print "%18.12f %11s pid=%s," %(time.time() - globalStartTime, current_process().name, current_process().pid), \
	"FINISHED:", datetime.datetime.now(), "at program run time %s " %(time.time() - globalStartTime), "seconds."
	print "%18.12f %11s pid=%s," %(time.time() - globalStartTime, current_process().name, current_process().pid), \
	"ALHAMDULILLAH WRY."

# =========================================================
"""
EXECUTION
===========================================================
wruslan@HP-ELBook8470p-ub1604-64b:~/Documents/temp1$ python
Python 2.7.12 (default, Dec  4 2017, 14:50:18) 
[GCC 5.4.0 20160609] on linux2
Type "help", "copyright", "credits" or "license" for more information.
>>> 
>>> import multiprocessing
>>> quit()

EXECUTION 1
===========================================================
wruslan@HP-ELBook8470p-ub1604-64b:~/Documents/temp1$ python 04-multi-processing.py 
0.000000953674 MainProcess pid=16602, BISMILLAH WRY
0.000068902969 MainProcess pid=16602, STARTING: 2018-08-05 19:15:59.476303 at program run time 0.000150918960571  seconds.
0.000164031982 MainProcess pid=16602, Actual computer cpu_count()  = 4 
0.000237941742 MainProcess pid=16602, Number of processes to create: 15
0.000259876251 MainProcess pid=16602, Begin multi-processing to run concurrent processes not concurrent threads.
0.026196002960 MainProcess pid=16602, Print from done_queue  =  ('0.245332956314   Process-7 pid=16610,', 'Finish child with result mult(7, 7) = 49')
0.246630907059 MainProcess pid=16602, Print from done_queue  =  ('0.261003971100   Process-3 pid=16606,', 'Finish child with result mult(1, 7) = 7')
0.261963844299 MainProcess pid=16602, Print from done_queue  =  ('0.282203912735   Process-2 pid=16605,', 'Finish child with result mult(2, 7) = 14')
0.283308029175 MainProcess pid=16602, Print from done_queue  =  ('0.290824890137   Process-4 pid=16607,', 'Finish child with result mult(3, 7) = 21')
0.291895866394 MainProcess pid=16602, Print from done_queue  =  ('0.378902912140   Process-6 pid=16609,', 'Finish child with result mult(4, 7) = 28')
0.380208015442 MainProcess pid=16602, Print from done_queue  =  ('0.403285026550   Process-5 pid=16608,', 'Finish child with result mult(5, 7) = 35')
0.404357910156 MainProcess pid=16602, Print from done_queue  =  ('0.487457036972   Process-9 pid=16612,', 'Finish child with result mult(9, 7) = 63')
0.488544940948 MainProcess pid=16602, Print from done_queue  =  ('0.494517803192  Process-10 pid=16613,', 'Finish child with result mult(6, 7) = 42')
0.495476961136 MainProcess pid=16602, Print from done_queue  =  ('0.508390903473   Process-1 pid=16604,', 'Finish child with result mult(0, 7) = 0')
0.509133815765 MainProcess pid=16602, Print from done_queue  =  ('0.521643877029  Process-13 pid=16616,', 'Finish child with result mult(8, 7) = 56')
0.523227930069 MainProcess pid=16602, Print from done_queue  =  ('0.587677001953  Process-14 pid=16617,', 'Finish child with result plus(3, 8) = 11')
0.588639020920 MainProcess pid=16602, Print from done_queue  =  ('0.634513854980  Process-11 pid=16614,', 'Finish child with result plus(2, 8) = 10')
0.635685920715 MainProcess pid=16602, Print from done_queue  =  ('0.718482971191   Process-8 pid=16611,', 'Finish child with result plus(1, 8) = 9')
0.719624996185 MainProcess pid=16602, Print from done_queue  =  ('0.930598974228  Process-15 pid=16618,', 'Finish child with result plus(4, 8) = 12')
0.931499004364 MainProcess pid=16602, Print from done_queue  =  ('0.957461833954  Process-12 pid=16615,', 'Finish child with result plus(0, 8) = 8')
0.958902835846 MainProcess pid=16602, FINISHED: 2018-08-05 19:16:00.435185 at program run time 0.959066867828  seconds.
0.959134817123 MainProcess pid=16602, ALHAMDULILLAH WRY.
wruslan@HP-ELBook8470p-ub1604-64b:~/Documents/temp1$ 

EXECUTION 2
===========================================================
wruslan@HP-ELBook8470p-ub1604-64b:~/Documents/temp1$ python 04-multi-processing.py  | sort
0.000000953674 MainProcess pid=17070, BISMILLAH WRY
0.000041961670 MainProcess pid=17070, STARTING: 2018-08-05 20:17:56.402341 at program run time 0.000102043151855  seconds.
0.000107049942 MainProcess pid=17070, Actual computer cpu_count()  = 4 
0.000159025192 MainProcess pid=17070, Number of processes to create: 15
0.000170946121 MainProcess pid=17070, Begin multi-processing to run concurrent processes not concurrent threads.
0.024323940277 MainProcess pid=17070, Print from done_queue  =  ('0.042391061783   Process-5 pid=17077,', 'Finish child with result mult(5, 7) = 35')
0.043257951736 MainProcess pid=17070, Print from done_queue  =  ('0.103610992432   Process-8 pid=17080,', 'Finish child with result mult(7, 7) = 49')
0.104824066162 MainProcess pid=17070, Print from done_queue  =  ('0.140308856964   Process-4 pid=17076,', 'Finish child with result mult(3, 7) = 21')
0.141304969788 MainProcess pid=17070, Print from done_queue  =  ('0.254593849182   Process-2 pid=17074,', 'Finish child with result mult(1, 7) = 7')
0.255899906158 MainProcess pid=17070, Print from done_queue  =  ('0.265964031219   Process-3 pid=17075,', 'Finish child with result mult(2, 7) = 14')
0.267159938812 MainProcess pid=17070, Print from done_queue  =  ('0.321465015411   Process-6 pid=17078,', 'Finish child with result mult(4, 7) = 28')
0.322818994522 MainProcess pid=17070, Print from done_queue  =  ('0.378835916519   Process-9 pid=17081,', 'Finish child with result mult(8, 7) = 56')
0.380008935928 MainProcess pid=17070, Print from done_queue  =  ('0.379969835281   Process-7 pid=17079,', 'Finish child with result mult(6, 7) = 42')
0.380954980850 MainProcess pid=17070, Print from done_queue  =  ('0.478909969330   Process-1 pid=17073,', 'Finish child with result mult(0, 7) = 0')
0.479954004288 MainProcess pid=17070, Print from done_queue  =  ('0.513936042786  Process-13 pid=17085,', 'Finish child with result mult(9, 7) = 63')
0.515058040619 MainProcess pid=17070, Print from done_queue  =  ('0.721731901169  Process-10 pid=17082,', 'Finish child with result plus(2, 8) = 10')
0.722784042358 MainProcess pid=17070, Print from done_queue  =  ('0.736804962158  Process-11 pid=17083,', 'Finish child with result plus(3, 8) = 11')
0.738076925278 MainProcess pid=17070, Print from done_queue  =  ('0.776344060898  Process-14 pid=17086,', 'Finish child with result plus(0, 8) = 8')
0.777623891830 MainProcess pid=17070, Print from done_queue  =  ('0.782939910889  Process-15 pid=17087,', 'Finish child with result plus(1, 8) = 9')
0.784394979477 MainProcess pid=17070, Print from done_queue  =  ('0.818614006042  Process-12 pid=17084,', 'Finish child with result plus(4, 8) = 12')
0.820370912552 MainProcess pid=17070, FINISHED: 2018-08-05 20:17:57.222743 at program run time 0.820557832718  seconds.
0.820574998856 MainProcess pid=17070, ALHAMDULILLAH WRY.
wruslan@HP-ELBook8470p-ub1604-64b:~/Documents/temp1$ 
"""
# =========================================================
\end{lstlisting}

