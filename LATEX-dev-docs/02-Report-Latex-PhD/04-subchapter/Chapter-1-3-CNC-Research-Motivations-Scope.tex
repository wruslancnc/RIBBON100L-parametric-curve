\justifying
\pagebreak
\section{Research Motivations}

\textbf{Commercial interest}
\vspace*{1\baselineskip}

Recall that CNC interpolation is the task that generates the actual reference commands that drives the CNC tool along the different axis-of-motions in the machine, such that, it accurately follows the desired machining path, in a timely and coordinated manner.
\vspace*{1\baselineskip}

On this fact, it can be said that interpolation is both the "brain and heart" of the CNC machine. The brain refers to the logic (program or algorithm) that gets the CNC running correctly and accurately, while the heart refers to the continuous, timely and coordinated supply of just the right power (pumping blood, oxygen and nutrients) to the CNC machine. Many people can copy, duplicate or reverse-engineer the physical aspects of the CNC machine because they are visible and physical, whereas the CNC interpolation software is intangible, and not easy to copy if properly protected. 
\vspace*{1\baselineskip}

\begin{tcolorbox}[colback=green!15!white,colframe=red!75!black,title=Research consideration no. 1]	
It was said that you can buy the machine, and the completed compiled software, but you cannot buy the source code that shows the internals on how the software was developed, so that you can learn from it. You need to develop one on your own and then learn from it.   
\end{tcolorbox}

% \pagebreak
\textbf{Ownership and accomplishment}
\vspace*{1\baselineskip}
	
We do not have our own CNC interpolator because existing interpolators are proprietary, meaning, owned by the CNC manufacturers and developed for their own unique commercial machines. It is plain common sense that manufacturers will not share or reveal their commercial secrets, except possibly marketing and generally well-known information. 
\vspace*{1\baselineskip}

We experienced real situations where suppliers interrogate us, the seller is suspicious of us, for example, regarding the purpose of our purchase when comes to high intellectual property products. Suppliers want to ensure that their products do not land into the wrong hands, particularly potential future competitors. They want to sell only to users but not knowledgeable developers. That is the game. The only way out is to always stay ahead, or always be the faster man than the fastest man.

\begin{tcolorbox}[colback=green!15!white,colframe=red!75!black,title=Research consideration no. 1]		
Therefore, it is of utmost interest for us to develop our own CNC interpolator. We know that in Malaysia, statistics show that more than 80 \% of commercial CNC machines and the like in use are imported. 
\end{tcolorbox}
	
\textbf{Enterpreneurship}
\vspace*{1\baselineskip}
	
We are interested and excited to manufacture our own commercial CNC machine in the near future, getting into the competitive CNC machining market and initially targeting at specific segments in our local market. In a borderless world today, we can even go to internet marketing, as many small and medium enterprises do today. We are excited means happy in the heart, for example, to study, explore, learn, undertake, execute and discover new things. 
\vspace*{1\baselineskip}
	
Our CNC product, interpolator plus machine, must be run effectively and efficiently. We must be effective means it must take effect, for example, to do exactly and correctly what it is supposed to do, that is according to its specifications. We must be efficient means be minimal in computation time and resources, for example, executes fast, speedy, compact and does not consume large resources.
\vspace*{1\baselineskip}

\textbf{Technical challenges}
\vspace*{1\baselineskip}
	
It is challenging technically, for example, to study the look-ahead control and feedrate filtering issues in CNC machining and come up with our way of addressing those issues.  Our success in providing alternative solutions to problems will arouse a personal sense of accomplishment, satisfaction combined with a feeling of worthiness, one who is useful and can contribute in some small way for the benefit of mankind. 
\vspace*{1\baselineskip}

\textbf{Future possibilities}
\vspace*{1\baselineskip}
	
With the exception of \textit{non-computable functions} in computing, software programming and control opens unlimited possibilities in what software technologies can do and accomplish. 
\vspace*{1\baselineskip}

The prospects extend from directing sequential step-by-step commands, parallel executions, sorting, search, scheduling, exotic software concepts like futures and asynchronous executions, communication channels, and so on, are all recent ideas coming from the human mind to automated logical reasoning, software self-healing, fuzzy reasoning, self-learning, and so on, conducted within and by the software itself. 
\vspace*{1\baselineskip}
	
It is not easy to think of doing things in parallel, executing many things in overlapping time, not necessarily concurrent, but that is reality of nature. It will be extremely satisfying, not only being able to run things in parallel, but doing so in true realtime elegantly, while meeting both the start and end design deadlines. The thoughts to come up with software solutions addressing those issues are truly exhilarating.    

% ==========================================
\clearpage
\pagebreak
\section{Scope of Research}

The scope of work proposed for this research study are as follows:
\begin{enumerate}
	\item Start from the provision of G-Codes, specifically RS274D NGC standard G-Code.
	\item Implement reference-pulse CNC interpolation for computational efficiency.
	\item Conduct look-ahead and feedrate error compensation in G-Code interpolation.
	\item Execute realtime, parallel, online/offline computations for the CNC control loop.
	\item Compare appropriate parallel execution methods for 2D/3D G-Code interpolation  
	\item Address designs for extension to 3-axis and 5-axis interpolation implementations.
\end{enumerate}

\section{Proposed Research Title}

The proposed research title shall be "A realtime and parallel look-ahead control and feedrate compensation strategy for  CNC reference-pulse interpolation." 

\section{Expected knowledge contributions}

The expected knowledge contributions for this research study are as follows:
\begin{enumerate}
	\item an implementation of a simple, practical and achievable strategy in CNC interpolation 
	\item a technique that utilizes both realtime and parallel execution in CNC machining
	\item a contribution in innovative ideas for an efficient design of the CNC interpolator 
\end{enumerate}

\pagebreak
%%% =========================================
\section{Summary on Introduction}
%% ==========================================

In this introductory chapter on our proposed research project, we began with a short overview of CNC, followed by the extensive adoption of software techniques in the control of CNC machines, particularly, contour tracking control executed in the CNC interpolation loop. 
\vspace*{1\baselineskip}
	
We briefly described recent designs in CNC machining environments, covering new cutting tools and technologies, and innovative software algorithms that address issues of path motion smoothness, machining speed control, and path error reduction.  
\vspace*{1\baselineskip}
	
The typical hardware and software requirements for a CNC machine were covered. A typical functional process flow for CNC machine operation was described. We illustrated what realtime, parallel and concurrent execution concepts mean to the common person compared to the correct understanding in software terminology through clear examples.
\vspace*{1\baselineskip}
	
We briefly explained the functions of various software components in a typical CNC system, the G-code interpreter, the G-Code interpolator, the motion controller, the error controller, the signal driver, service controller, the human-machine controller and the overall CNC controller loop. 
\vspace*{1\baselineskip}

The interface and boundary point between software and hardware in CNC machine was specially defined, by differentiating the terms signal driver software against signal driver hardware. The signal driver hardware is generally referred to as electrical pulse generating devices. 
\vspace*{1\baselineskip}

Some signal driver hardware interface boards that have been used in our previous research projects were listed. In addition, the issues that need to be addressed by each of the software components to work with those signal driver devices were also discussed.
\vspace*{1\baselineskip}
	
On the hardware side of the CNC machine, we described the industry practice of a compatible set of motor-driver and electric-motor pair, such that there is no mismatch of performance, for example, pulse frequency handling. In standard practice, the software control loop of the CNC machine only interacts with the motor-driver hardware, not directly to the electric-motor. We shared survey results on commercial companies with leading products in the CNC machining market, and identified the major brands that offer high end and low end CNC machines.   
\vspace*{1\baselineskip}
	
Finally, we described our motivations for conducting research on CNC machines, our proposed research topic, our proposed the scope of research and the expected contributions of the work. 
\vspace*{1\baselineskip}

In summary, the focus of our research is CNC interpolation. We restate, that CNC interpolation is about the task of generating and sending signal pulses to the respective electric motors at the machine's axis-of-motions, in a coordinated and timely manner, such that the machine tool accurately tracks the desired contour path in the move from the current position to the next position. To achieve this task, the interpolator must generate the right number of signal pulses that achieves the desired distance for the  move, and the correct pulse feedrate (frequency) that achieves the desired velocity for the move.    

% ============================================

